%%%% CAPÍTULO 1 - INTRODUÇÃO
%%
%% Deve apresentar uma visão global da pesquisa, incluindo: breve histórico, importância e justificativa da escolha do tema,
%% delimitações do assunto, formulação de hipóteses e objetivos da pesquisa e estrutura do trabalho.

%% Título e rótulo de capítulo (rótulos não devem conter caracteres especiais, acentuados ou cedilha)
\chapter{Introdução}
\label{cap:introducao}

No cenário dinâmico do desenvolvimento web, as ferramentas de desenvolvimento (\textit{devtools}) integradas aos navegadores se tornaram indispensáveis para a criação e manutenção de aplicações web complexas e interativas. Essas ferramentas, acessíveis mediante um simples atalho de teclado (geralmente na tecla F12) \cite{firefox}, oferecem aos desenvolvedores um conjunto robusto de funcionalidades que facilitam a depuração de código, a inspeção do DOM, a análise de desempenho e diversas outras tarefas importantes para a construção de experiências digitais de alta qualidade \cite{chrome}.

Este estudo pretende realizar uma análise exploratória das principais ferramentas de desenvolvimento presentes nos navegadores web mais utilizados atualmente, sendo eles o Chrome, Firefox, Edge e Safari \footnote{https://gs.statcounter.com/browser-market-share\#monthly-202401-202501}. Por meio de uma análise de suas funcionalidades, interfaces e aplicações práticas, busca-se compreender como essas ferramentas influenciam o dia a dia dos desenvolvedores e quais são os benefícios que elas proporcionam para o desenvolvimento de aplicações web.\\

Além de comparar as funcionalidades e interfaces das \textit{devtools} dos principais navegadores, este estudo também busca entender como os desenvolvedores utilizam essas ferramentas em seus projetos. Por meio de uma pesquisa com profissionais da área, será possível identificar as práticas mais comuns, as dificuldades encontradas e as necessidades não atendidas. Ao final, espera-se que este estudo contribua para o avanço do conhecimento sobre as \textit{devtools} e para a melhoria das práticas de desenvolvimento web, tornando o trabalho dos desenvolvedores mais eficiente e produtivo.


\section{Objetivos}
\label{sec:objetivos}

Em sequida será apresentado o objetivo geral do trabalho e em seguida os objetivos específicos.

\section{Objetivo Geral}
Este estudo visa fornecer uma análise aprofundada das ferramentas de desenvolvimento (\textit{devtools}) integradas aos principais navegadores web. Por meio de uma análise exploratória de suas funcionalidades, interfaces e aplicações práticas, busca-se compreender como essas ferramentas influenciam o dia a dia dos desenvolvedores e quais são os benefícios que elas proporcionam para o desenvolvimento de aplicações web. 

\section{Objetivos especificos}
\begin{itemize}
    \item Comparar as funcionalidades oferecidas pelos \textit{devtools} dos principais navegadores da atualidade, identificando semelhanças, diferenças e características únicas de cada ferramenta.
    \item Identificar as práticas mais comuns de utilização das \textit{devtools} entre os desenvolvedores, incluindo as ferramentas e recursos mais utilizados.
    \item Avaliar o impacto do uso das \textit{devtools} na produtividade dos desenvolvedores, na correção de erros e na qualidade final dos sistemas.
    \item Avaliar a necessidade de plugins e extensões durante o desenvolvimento de sistemas web.
\end{itemize}


\section{Justificativa}

A crescente demanda por soluções digitais e a evolução constante da internet impulsionam o desenvolvimento de aplicações web cada vez mais sofisticadas e complexas. Nesse contexto, as ferramentas de desenvolvimento web desempenham um papel fundamental, permitindo que os desenvolvedores criem experiências online inovadoras e eficientes. No entanto, a diversidade de navegadores disponíveis no mercado e a constante atualização de suas funcionalidades tornam a escolha da ferramenta ideal uma tarefa desafiadora.

\section{Relevância do desenvolvimento web}
A relevância do desenvolvimento web transcende os limites da área tecnológica, impactando diretamente a sociedade na totalidade. A internet se tornou uma plataforma indispensável para comunicação, comércio, educação e entretenimento, e as aplicações web são os alicerces que sustentam essa infraestrutura digital \cite{tecnologia-web}. A compreensão das ferramentas e técnicas utilizadas nesse processo é crucial para acompanhar a evolução tecnológica e atender às demandas de um mercado cada vez mais exigente.

\section{Contribuição com a comunidade}
Este trabalho visa contribuir para a comunidade de desenvolvedores web ao fornecer uma análise exploratória dos principais navegadores disponíveis no mercado. Ao identificar as vantagens e desvantagens de cada ferramenta, o estudo poderá auxiliar os desenvolvedores a tomar decisões mais informadas na escolha da ferramenta ideal para seus projetos. Além disso, a pesquisa poderá identificar tendências e direcionar futuras pesquisas na área de desenvolvimento web, impulsionando a evolução da área.




\section{Estrutura do trabalho}
\label{sec:estruturaTrabalho}

