% CAPA---------------------------------------------------------------------------------------------------

% ORIENTAÇÕES GERAIS-------------------------------------------------------------------------------------
% Caso algum dos campos não se aplique ao seu trabalho, como por exemplo,
% se não houve coorientador, apenas deixe vazio.
% Exemplos: 
% \coorientador{}
% \departamento{}

% DADOS DO TRABALHO--------------------------------------------------------------------------------------
\titulo{FERRAMENTAS DE DESENVOLVIMENTO WEB: UMA DOCUMENTAÇÃO TÉCNICA}
\autor{Lucas Pereira Machado}
\local{Toledo}
%\data{2025} % \data must be set in the document preamble (main .tex); keep commented here to avoid the error.
\newcommand{\anoTrabalho}{2025} % use \anoTrabalho within this file if you need the year locally

% NATUREZA DO TRABALHO-----------------------------------------------------------------------------------
\projeto{Proposta de Trabalho de Conclusão de Curso}

% TÍTULO ACADÊMICO---------------------------------------------------------------------------------------
% - Bacharel ou Tecnólogo
\tituloAcademico{Tecnólogo}

% DADOS DA INSTITUIÇÃO-----------------------------------------------------------------------------------
% Coloque o nome do curso de graduação em "programa"
% Formato para o logo da Instituição: \logoinstituicao{<escala>}{<caminho/nome do arquivo>}
\instituicao{Universidade Tecnológica Federal do Paraná}
\departamento{Câmpus Toledo}
\programa{COTSI - Curso de Tecnologia em Sistemas para Internet}
\logoinstituicao{0.2}{dados/figuras/logo-instituicao.png} 

% DADOS DOS ORIENTADORES---------------------------------------------------------------------------------
\orientador{Prof. Mr. Eduardo Pezutti Beletato dos Santos}
%\orientador[Orientadora:]{Nome da orientadora}
\instOrientador{}
