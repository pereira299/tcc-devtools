\chapter{Considerações Finais}

Este estudo cumpriu seu objetivo de analisar, de forma exploratória, as ferramentas de desenvolvimento 
(devtools) integradas aos principais navegadores do mercado, trazendo um panorama geral dessas 
ferramentas. Por meio deste trabalho, ficou claro que, embora todas as plataformas compartilhem um 
conjunto de recursos para a manipulação do DOM e depuração de scripts, cada navegador apresenta 
particularidades distintas, adaptando-se a diferentes demandas do fluxo de trabalho em desenvolvimento 
web.

Na análise comparativa das funcionalidades, o Firefox distinguiu-se pelos recursos focados em 
privacidade e design (como CSS Grid e Flexbox), enquanto o Microsoft Edge demonstrou robustez na 
integração com o ecossistema Windows e o VS Code. Por sua vez, o Safari Web Inspector, mesmo restrito 
ao ambiente Apple, provou-se indispensável para a depuração e auditoria de performance em dispositivos 
iOS e macOS.

Os dados obtidos na pesquisa de campo revelaram um cenário relevante sobre as práticas de utilização. 
Verificou-se que, a despeito da sofisticação das devtools modernas, existe uma subutilização de seus 
recursos avançados. O fato de apenas 27\% dos desenvolvedores dominarem a totalidade das ferramentas do 
Chrome DevTools, restringindo-se majoritariamente ao uso do "Inspetor de Elementos" e do "Console", 
aponta para uma lacuna significativa na capacitação técnica dos profissionais.

Em relação à produtividade e à qualidade de software, conclui-se que as devtools são vitais para a 
detecção precoce de erros e a otimização de performance (via Lighthouse e painéis de desempenho). 
Contudo, a eficácia desses recursos é limitada pelo nível de conhecimento do usuário. Observou-se, 
ainda, que desenvolvedores seniores tendem a complementar o uso do navegador com plugins de IDE, 
sugerindo que, para arquiteturas complexas, as ferramentas nativas, isoladamente, ainda não suprem 
todas as necessidades de desenvolvimento.

Por fim, este estudo sugere que a evolução do desenvolvimento web não depende apenas de novas 
funcionalidades nos navegadores, mas sim da disseminação do conhecimento sobre as ferramentas já 
existentes. Recomenda-se, para trabalhos futuros, a criação de materiais didáticos focados nas 
ferramentas de "Memória", "Desempenho" e "Segurança", áreas que demonstraram maior carência de domínio 
técnico, visando elevar o padrão de qualidade e eficiência na construção de aplicações web.