\section{Materiais utilizados}
\label{sec:materiais}

As ferramentas de desenvolvimento do navegador, também conhecidas como \textit{DevTools}, são conjuntos de utilitários incorporados nos navegadores da web, como Google Chrome e Mozilla Firefox. Esses recursos permitem que os desenvolvedores inspecionem e depurem o código-fonte, analisem o desempenho e verifiquem o uso de memória de seus aplicativos \cite{apple}. Durante o desenvolvimento deste trabalho, foram utilizadas as principais ferramentas disponíveis no mercado, conforme detalhado a seguir.
\begin{figure}[!htb]
    \centering
    \caption{Chrome Devtools}
    \includegraphics[width=0.65 \linewidth]{assets/chrome-devtools.png}\\
    {\footnotesize Fonte: Chrome for Developers}
    \label{fig:chrome-devtools}
\end{figure}

Inicialmente, destaca-se o Chrome DevTools (Figura \ref{fig:chrome-devtools}). Este representa um conjunto abrangente de ferramentas integradas ao navegador Google Chrome, essenciais para o desenvolvimento e depuração de aplicações web. Com ele, é possível inspecionar o DOM, analisar o desempenho da página, depurar JavaScript e simular diferentes dispositivos \cite{chrome}.
\begin{figure}[!htb]
    \centering
    \caption{Firefox Developer Tools}
    \includegraphics[width=0.55\linewidth]{assets/firefox.png}\\
    {\footnotesize Fonte: Firefox Source Tree Documentation}
    \label{fig:firefox}
\end{figure}
Como uma alternativa robusta ao ambiente da Google, o Firefox Developer Tools (Figura \ref{fig:firefox}) apresenta um foco maior em privacidade e personalização. Essas ferramentas proporcionam um ambiente de desenvolvimento completo, permitindo a inspeção de elementos, a análise de redes, a depuração de JavaScript, a visualização de estilos CSS e a otimização de desempenho. Ademais, o Firefox Developer Tools integra-se ao ecossistema Mozilla, incluindo o editor Visual Studio Code, o que facilita o fluxo de trabalho dos desenvolvedores \cite{firefox}.

\begin{figure}[!htb]
    \centering
    \caption{Edge DevTools}
    \includegraphics[width=0.65\linewidth]{assets/edge.png}\\
    {\footnotesize Fonte: Microsoft Learn}
    \label{fig:edge}
\end{figure}

No mesmo cenário de navegadores modernos, os Edge DevTools (Figura \ref{fig:edge}), incorporados ao Microsoft Edge, oferecem uma experiência eficiente com uma interface visualmente atraente. Apresentando funcionalidades similares às do Chrome, esta ferramenta permite inspecionar elementos, depurar JavaScript e analisar o desempenho da página. A sua forte integração com o Microsoft Visual Studio Code torna os Edge DevTools uma opção particularmente atraente para desenvolvedores que utilizam a plataforma Windows \cite{edge}.

Por fim, o Safari Web Inspector (Figura \ref{fig:safari}) é a ferramenta de desenvolvimento integrada ao navegador Apple Safari, projetada para oferecer uma experiência de depuração de alta qualidade. Com foco em desempenho e usabilidade, o Web Inspector permite inspecionar elementos, depurar JavaScript e simular diferentes dispositivos dentro do ecossistema Apple \cite{apple}.