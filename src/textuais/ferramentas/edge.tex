\section{Microsoft Edge}

\subsection{Elementos}
\begin{figure}[!htb]
    \centering
    \caption{Ferramenta "Elementos" do Microsoft Edge Devtools}
    \includegraphics[width=0.8\linewidth]{assets/tools/edge-elements.png}\\
    {\footnotesize Fonte: Microsoft Learn}
    \label{fig:edge-elements}
\end{figure}
O painel \textbf{Elementos} (Elements), exibido na figura \ref{fig:edge-elements}, oferece uma interface robusta para a inspeção e manipulação do Document Object 
Model (DOM) e das Folhas de Estilo em Cascata (CSS) em tempo real. Ele proporciona uma representação interativa da árvore 
DOM, refletindo o estado dinâmico da aplicação, que pode divergir do HTML fonte original devido a manipulações via JavaScript \cite{edge-elements}. 
A ferramenta facilita a depuração de layout através da edição direta de atributos HTML, conteúdo textual e propriedades CSS, 
permitindo ainda a simulação de pseudo-estados (ex: \textit{:hover}) e a análise do modelo de caixa (box model) \cite{edge-elements}.

\subsection{Console}
\begin{figure}[!htb]
    \centering
    \caption{Ferramenta "Console" do Microsoft Edge Devtools}
    \includegraphics[width=0.8\linewidth]{assets/tools/edge-console.png}\\
    {\footnotesize Fonte: Microsoft Learn}
    \label{fig:edge-console}
\end{figure}
O \textbf{Console} (Console), apresentado na figura \ref{fig:edge-console}, atua como um ambiente interativo de Read-Eval-Print Loop (REPL) para a execução de scripts 
JavaScript. Sua função primária é o diagnóstico de tempo de execução, servindo como o principal canal para o registro 
(\textit{logging}) de erros, advertências e saídas diagnósticas (ex: \texttt{console.log()}) \cite{edge-console}. Permite a interação direta 
com o contexto da página, viabilizando a inspeção e manipulação programática do DOM e do objeto \textit{Window}, além do 
teste de expressões e a utilização de funções utilitárias de depuração \cite{edge-console}.

\subsection{Fontes}
\begin{figure}[!htb]
    \centering
    \caption{Ferramenta "Fontes" do Microsoft Edge Devtools}
    \includegraphics[width=0.8\linewidth]{assets/tools/edge-sources.png}\\
    {\footnotesize Fonte: Microsoft Learn}
    \label{fig:edge-sources}
\end{figure}
O painel \textbf{Fontes} (Sources), ilustrado na figura \ref{fig:edge-sources}, funciona como um ambiente de depuração de código-fonte integrado. É projetado para a 
análise, edição e depuração de JavaScript front-end. A ferramenta permite a navegação pelos recursos da página, a edição 
de scripts e a utilização de um depurador de JavaScript \cite{edge-sources}. Este último suporta funcionalidades essenciais como a definição 
de pontos de interrupção (\textit{breakpoints}), a execução passo a passo (\textit{step-through}) do código, a inspeção 
da pilha de chamadas (\textit{call stack}) e a monitorização de variáveis em escopo \cite{edge-sources}.

\subsection{Rede}
\begin{figure}[!htb]
    \centering
    \caption{Ferramenta "Rede" do Microsoft Edge Devtools}
    \includegraphics[width=0.8\linewidth]{assets/tools/edge-network.png}\\
    {\footnotesize Fonte: Microsoft Learn}
    \label{fig:edge-network}
\end{figure}
A ferramenta \textbf{Rede} (Network),  ilustrada na figura \ref{fig:edge-network}, é um componente de diagnóstico focado na análise do tráfego de rede. Ela monitora e 
registra todas as solicitações (requests) e respostas (responses) HTTP/S, incluindo documentos, scripts, folhas de estilo, 
mídias e chamadas de API (XHR/Fetch) \cite{edge-network}. A ferramenta oferece uma análise detalhada de cabeçalhos, códigos de status, conteúdo 
e temporização (através de um gráfico de \textit{Waterfall}), sendo crucial para a identificação de gargalos de latência, 
falhas de requisição e otimização de \textit{payloads} \cite{edge-network}. Permite ainda a simulação de condições de rede adversas 
(\textit{throttling}) \cite{edge-network}.

\subsection{Desempenho}
\begin{figure}[!htb]
    \centering
    \caption{Ferramenta "Desempenho" do Microsoft Edge Devtools}
    \includegraphics[width=0.8\linewidth]{assets/tools/edge-performance.png}\\
    {\footnotesize Fonte: Microsoft Learn}
    \label{fig:edge-performance}
\end{figure}
O painel \textbf{Desempenho} (Performance), como exibido na figura \ref{fig:edge-performance}, é um utilitário de perfilamento (\textit{profiling}) avançado para a análise da 
performance de tempo de execução. A ferramenta opera em duas modalidades: um monitoramento em tempo real dos Core Web Vitals 
(LCP, CLS, INP) e a gravação de um perfil detalhado \cite{edge-performance}. Este perfil captura uma linha do tempo da atividade da \textit{thread} 
principal, incluindo a execução de JavaScript, operações de renderização (\textit{layout} e \textit{paint}), e eventos de 
interação \cite{edge-performance}. A análise deste traço permite a identificação precisa de gargalos de CPU e operações de longa duração que afetam a 
responsividade da aplicação .

\subsection{Aplicação}
\begin{figure}[!htb]
    \centering
    \caption{Ferramenta "Aplicação" do Microsoft Edge Devtools}
    \includegraphics[width=0.8\linewidth]{assets/tools/edge-application.png}\\
    {\footnotesize Fonte: Microsoft Learn}
    \label{fig:edge-application}
\end{figure}
A ferramenta \textbf{Aplicação} (Application), apresentado na figura \ref{fig:edge-application}, proporciona uma interface centralizada para a inspeção e gerenciamento do 
armazenamento no lado do cliente. Ela permite a análise e manipulação de mecanismos de armazenamento como Local Storage, 
Session Storage, IndexedDB, Cookies e Cache Storage \cite{edge-application}. Adicionalmente, pode ser utilizado para a depuração de Progressive 
Web Apps (PWAs), oferecendo controle sobre o ciclo de vida dos \textit{Service Workers}, a validação do Manifesto da Aplicação 
e a inspeção de serviços em segundo plano (\textit{Background Services}) \cite{edge-application}.

\subsection{Exibição 3D}
\begin{figure}[!htb]
    \centering
    \caption{Ferramenta "Exibição 3D" do Microsoft Edge Devtools}
    \includegraphics[width=0.8\linewidth]{assets/tools/edge-3d-view.png}\\
    {\footnotesize Fonte: Microsoft Learn}
    \label{fig:edge-3d-view}
\end{figure}
A \textbf{Exibição 3D} (3D View), ilustrada na figura \ref{fig:edge-3d-view},  é um recurso de visualização avançado para a depuração de contextos de renderização complexos. 
Ao modelar a página em um espaço tridimensional, a ferramenta facilita a identificação de problemas de sobreposição de elementos 
(z-index), a análise da hierarquia do DOM e a inspeção das camadas de composição (\textit{composited layers}) criadas pelo motor 
de renderização \cite{edge-3d-view}. A ferramenta segmenta a visualização em abas dedicadas para Camadas Compostas, Z-index e DOM, auxiliando na 
compreensão da estrutura de renderização \cite{edge-3d-view}.

\subsection{Animações}
\begin{figure}[!htb]
    \centering
    \caption{Ferramenta "Animações" do Microsoft Edge Devtools}
    \includegraphics[width=0.8\linewidth]{assets/tools/edge-animations.png}\\
    {\footnotesize Fonte: Microsoft Learn}
    \label{fig:edge-animations}
\end{figure}
O inspetor de \textbf{Animações} (Animation Inspector), exibido na figura \ref{fig:edge-animations}, é um utilitário focado na inspeção e depuração de animações declarativas. 
A ferramenta captura sequências de animação (CSS Animations, CSS Transitions e Web Animations), agrupando-as com base em seu 
tempo de início \cite{edge-animations}. Permite a desaceleração, repetição e análise do código-fonte das animações, além de possibilitar a modificação 
interativa de parâmetros como duração, atraso (\textit{delay}) e temporização de \textit{keyframes} \cite{edge-animations}.

\subsection{Alterações}
\begin{figure}[!htb]
    \centering
    \caption{Ferramenta "Alterações" do Microsoft Edge Devtools}
    \includegraphics[width=0.8\linewidth]{assets/tools/edge-changes.png}\\
    {\footnotesize Fonte: Microsoft Learn}
    \label{fig:edge-changes}
\end{figure}
A ferramenta \textbf{Alterações} (Changes), apresentada na figura \ref{fig:edge-changes}, monitora modificações efetuadas nos arquivos-fonte (CSS, JavaScript, HTML) 
diretamente no ambiente de desenvolvimento. Sua função primária é apresentar um comparativo visual (\textit{diff}) que 
detalha as discrepâncias (adições e remoções de linhas) entre o estado original do recurso e as edições locais \cite{edge-changes}. Isso facilita 
a revisão das alterações antes de sua persistência no sistema de arquivos, permitindo também a reversão das modificações \cite{edge-changes}.

\subsection{Cobertura}
\begin{figure}[!htb]
    \centering
    \caption{Ferramenta "Cobertura" do Microsoft Edge Devtools}
    \includegraphics[width=0.8\linewidth]{assets/tools/edge-coverage.png}\\
    {\footnotesize Fonte: Microsoft Learn}
    \label{fig:edge-coverage}
\end{figure}
O painel \textbf{Cobertura} (Coverage), exibido na figura \ref{fig:edge-coverage}, é um utilitário analítico que quantifica a utilização de código JavaScript e CSS 
durante o carregamento e a interação com a página. A ferramenta identifica segmentos de código não executados, fornecendo 
métricas de bytes (utilizados vs. não utilizados) e um detalhamento visual linha a linha \cite{edge-coverage}. Esta análise é fundamental para 
otimizações de \textit{tree-shaking} e remoção de código morto (\textit{dead code}), visando a redução do \textit{payload} 
da aplicação e a melhoria do tempo de carregamento \cite{edge-coverage}.

\subsection{Analisador de Falhas}
\begin{figure}[!htb]
    \centering
    \caption{Ferramenta "Analisador de Falhas" do Microsoft Edge Devtools}
    \includegraphics[width=0.8\linewidth]{assets/tools/edge-crash-analyzer.png}\\
    {\footnotesize Fonte: Microsoft Learn}
    \label{fig:edge-crash-analyzer}
\end{figure}
O \textbf{Analisador de Falhas} (Crash Analyzer), ilustrado na figura \ref{fig:edge-crash-analyzer}, é um utilitário de diagnóstico pós-morte concebido para analisar falhas de 
aplicações em produção. Sua principal função é processar \textit{stack traces} (pilhas de chamadas) de JavaScript minificados, 
que são frequentemente ofuscados \cite{edge-crash-analyzer}. Ao aplicar os \textit{source maps} correspondentes, a ferramenta reverte o código ao seu 
estado original não minificado, permitindo a identificação da causa raiz da falha ao mapear o erro de volta aos nomes de funções 
e linhas de código legíveis \cite{edge-crash-analyzer}.

\subsection{Descrição Geral de CSS}
\begin{figure}[!htb]
    \centering
    \caption{Ferramenta "Descrição Geral de CSS" do Microsoft Edge Devtools}
    \includegraphics[width=0.8\linewidth]{assets/tools/edge-css-overview.png}\\
    {\footnotesize Fonte: Microsoft Learn}
    \label{fig:edge-css-overview}
\end{figure}
A ferramenta \textbf{Descrição Geral de CSS} (CSS Overview), apresentada na figura \ref{fig:edge-css-overview}, 
realiza uma auditoria estática do CSS de uma página. Ela captura um instantâneo do estado do CSS e gera um relatório 
que cataloga o uso de cores, tipografia (\textit{fonts}) e \textit{media queries} \cite{edge-css-overview}. A ferramenta 
é projetada para identificar inconsistências no design system, como cores duplicadas ou estilos de fonte não padronizados, 
e destaca problemas de acessibilidade, notadamente questões de contraste de cor insuficientes \cite{edge-css-overview}.

\subsection{Problemas}
O painel \textbf{Problemas} (Issues) funciona como um agregador proativo de diagnósticos. Ele analisa continuamente a página e 
consolida problemas detectados em categorias como acessibilidade, compatibilidade entre navegadores 
(\textit{cross-browser compatibility}), desempenho, segurança e conformidade com PWA \cite{edge-issues}. A ferramenta fornece descrições 
contextuais de cada problema e links diretos para a documentação ou para o recurso afetado, centralizando o feedback de 
múltiplas fontes de auditoria \cite{edge-issues}.

\subsection{Farol}
\begin{figure}[!htb]
    \centering
    \caption{Ferramenta "Farol" do Microsoft Edge Devtools}
    \includegraphics[width=0.8\linewidth]{assets/tools/edge-lighthouse.png}\\
    {\footnotesize Fonte: Microsoft Learn}
    \label{fig:edge-lighthouse}
\end{figure}
A ferramenta \textbf{Lighthouse} (anteriormente "Audits"), exibida na figura \ref{fig:edge-lighthouse}, é um utilitário de auditoria automatizada para a avaliação da 
qualidade de páginas web. Ela executa uma série de testes e gera relatórios de diagnóstico abrangentes, pontuando a página em 
métricas de Desempenho, Acessibilidade, Melhores Práticas, SEO e Progressive Web App (PWA) \cite{edge-lighthouse}. Os relatórios fornecem uma linha de 
base quantificável e recomendações acionáveis para a otimização de cada uma dessas categorias \cite{edge-lighthouse}.

\subsection{Mídia}
\begin{figure}[!htb]
    \centering
    \caption{Ferramenta "Mídia" do Microsoft Edge Devtools}
    \includegraphics[width=0.8\linewidth]{assets/tools/edge-media.png}\\
    {\footnotesize Fonte: Microsoft Learn}
    \label{fig:edge-media}
\end{figure}
O painel \textbf{Mídia} (Media), apresentado na figura \ref{fig:edge-media}, é um utilitário de depuração específico para a inspeção de reprodutores de mídia 
(\texttt{<video>} e \texttt{<audio>}) em uma página. A ferramenta monitora e exibe propriedades do reprodutor, eventos do ciclo 
de vida da mídia (ex: \textit{play}, \textit{pause}, \textit{seek}) e mensagens de log \cite{edge-media}. Facilita a depuração de problemas de 
reprodução, \textit{buffering} e eventos de mídia \cite{edge-media}.

\subsection{Inspetor de Memória}
\begin{figure}[!htb]
    \centering
    \caption{Ferramenta "Inspetor de Memória" do Microsoft Edge Devtools}
    \includegraphics[width=0.8\linewidth]{assets/tools/edge-memory-inspector.png}\\
    {\footnotesize Fonte: Microsoft Learn}
    \label{fig:edge-memory-inspector}
\end{figure}
O \textbf{Inspetor de Memória} (Memory Inspector), ilustrado na figura \ref{fig:edge-memory-inspector}, é um utilitário para a 
inspeção de baixo nível de buffers de memória binária. Ele é projetado para analisar \textit{ArrayBuffer}, \textit{TypedArray}, \textit{DataView} e, crucialmente, a memória linear 
de WebAssembly (Wasm) \cite{edge-memory-inspector}. A ferramenta exibe os bytes brutos em formato hexadecimal, uma representação ASCII adjacente e um 
inspetor de valores que interpreta os dados em múltiplos formatos (ex: inteiros de 32 bits, \textit{floats}), suportando a 
alternância entre \textit{big-endian} e \textit{little-endian} \cite{edge-memory-inspector}.

\subsection{Condições de Rede}
\begin{figure}[!htb]
    \centering
    \caption{Ferramenta "Condições de Rede" do Microsoft Edge Devtools}
    \includegraphics[width=0.8\linewidth]{assets/tools/edge-network-conditions.png}\\
    {\footnotesize Fonte: Microsoft Learn}
    \label{fig:edge-network-conditions}
\end{figure}
A ferramenta \textbf{Condições de Rede} (Network Conditions), apresentada na figura \ref{fig:edge-network-conditions}, é um utilitário de simulação de ambiente. Suas funções primárias 
incluem a desativação do cache do navegador, a aplicação de limitação de banda (\textit{throttling}) para emular diferentes 
velocidades de conexão (ex: 3G lento) e a capacidade de sobrescrever a \textit{string} de agente do usuário 
(\textit{user agent}) \cite{edge-network-conditions}. Adicionalmente, permite a configuração das codificações de conteúdo (Content-Encodings) aceitas para 
testar o processamento de respostas comprimidas \cite{edge-network-conditions}.

\subsection{Console de Rede}
\begin{figure}[!htb]
    \centering
    \caption{Ferramenta "Console de Rede" do Microsoft Edge Devtools}
    \includegraphics[width=0.8\linewidth]{assets/tools/edge-network-console.png}\\
    {\footnotesize Fonte: Microsoft Learn}
    \label{fig:edge-network-console}
\end{figure}
O \textbf{Console de Rede} (Network Console), exibido na figura \ref{fig:edge-network-console}, é um cliente de API integrado, projetado para a composição, envio e teste de 
solicitações HTTP. Permite a configuração detalhada de requisições, especificando o método (GET, POST, etc.), URL, cabeçalhos 
e corpo da solicitação \cite{edge-network-console}. É usado primariamente para a depuração de APIs Web (REST/OpenAPI) e é compatível com a importação e 
exportação de coleções (ex: formato Postman) \cite{edge-network-console}.

\subsection{Bloqueio de Solicitações de Rede}
\begin{figure}[!htb]
    \centering
    \caption{Ferramenta "Bloqueio de Solicitações de Rede" do Microsoft Edge Devtools}
    \includegraphics[width=0.8\linewidth]{assets/tools/edge-network-request-blocking.png}\\
    {\footnotesize Fonte: Microsoft Learn}
    \label{fig:edge-network-request-blocking}
\end{figure}
Este utilitário, apresentado na figura \ref{fig:edge-network-request-blocking}, permite a simulação de falhas de rede ao bloquear o carregamento de recursos específicos. Os desenvolvedores 
podem definir padrões (incluindo \textit{wildcards}) para impedir que solicitações de rede para URLs, domínios ou tipos de 
arquivos específicos sejam concluídas \cite{edge-network-blocking}. Esta funcionalidade é essencial para testar a resiliência da aplicação, o comportamento 
de \textit{fallback} e a renderização da página em cenários de indisponibilidade de recursos críticos \cite{edge-network-blocking}.

\subsection{Monitoramento de Desempenho}
\begin{figure}[!htb]
    \centering
    \caption{Ferramenta "Monitoramento de Desempenho" do Microsoft Edge Devtools}
    \includegraphics[width=0.8\linewidth]{assets/tools/edge-performance-monitor.png}\\
    {\footnotesize Fonte: Microsoft Learn}
    \label{fig:edge-performance-monitor}
\end{figure}
O \textbf{Monitoramento de Desempenho} (Performance Monitor), ilustrado na figura \ref{fig:edge-performance-monitor}, oferece uma visualização em tempo real de métricas de desempenho 
de tempo de execução. Diferente do perfilamento detalhado do painel "Desempenho", este monitor exibe gráficos contínuos de 
indicadores-chave, como uso da CPU, tamanho do \textit{heap} de JavaScript, contagem de nós DOM e a frequência de recálculos de 
layout e estilo por segundo, auxiliando na identificação de consumo excessivo de recursos \cite{edge-performance-monitor}.

\subsection{Gravador}
\begin{figure}[!htb]
    \centering
    \caption{Ferramenta "Gravador" do Microsoft Edge Devtools}
    \includegraphics[width=0.8\linewidth]{assets/tools/edge-recorder.png}\\
    {\footnotesize Fonte: Microsoft Learn}
    \label{fig:edge-recorder}
\end{figure}
A ferramenta \textbf{Gravador} (Recorder), apresentado na figura \ref{fig:edge-recorder}, permite a captura e reprodução de fluxos de interação do usuário. Ela registra 
ações como cliques, entradas de teclado e eventos de navegação, permitindo que a sequência seja executada automaticamente \cite{edge-recorder}. 
Este utilitário é usado para automatizar testes de regressão, analisar fluxos de usuário complexos e medir o desempenho de 
tempo de execução durante a reprodução do fluxo, identificando gargalos de performance em interações específicas \cite{edge-recorder}.

\subsection{Renderização}
\begin{figure}[!htb]
    \centering
    \caption{Ferramenta "Renderização" do Microsoft Edge Devtools}
    \includegraphics[width=0.8\linewidth]{assets/tools/edge-rendering.png}\\
    {\footnotesize Fonte: Microsoft Learn}
    \label{fig:edge-rendering}
\end{figure}
O painel \textbf{Renderização} (Rendering), exibido na figura \ref{fig:edge-rendering}, é um conjunto de utilitários de diagnóstico focados na visualização da saída do 
motor de renderização. Suas funcionalidades incluem a emulação de recursos de mídia CSS (como \texttt{prefers-color-scheme} 
e modo de impressão), a simulação de deficiências visuais (ex: daltonismo, visão turva) e a ativação de sobreposições de 
depuração (ex: \textit{paint flashing}, \textit{layout shift regions}) \cite{edge-rendering}.

\subsection{Pesquisa}
A ferramenta \textbf{Pesquisa} (Search), ilustrada na figura \ref{fig:edge-search}, oferece uma funcionalidade de busca global que abrange todos os arquivos de recursos 
carregados pela página (HTML, CSS, JS). Ela suporta a localização de sequências de texto literais e expressões regulares, com 
opções de sensibilidade a maiúsculas e minúsculas \cite{edge-search}. Os resultados são listados por arquivo, e a seleção de um resultado direciona 
o usuário para a linha correspondente no painel "Fontes" \cite{edge-search}.

\subsection{Segurança}
\begin{figure}[!htb]
    \centering
    \caption{Ferramenta "Segurança" do Microsoft Edge Devtools}
    \includegraphics[width=0.8\linewidth]{assets/tools/edge-security.png}\\
    {\footnotesize Fonte: Microsoft Learn}
    \label{fig:edge-security}
\end{figure}
O painel \textbf{Segurança} (Security), apresentado na figura \ref{fig:edge-security}, analisa o status de segurança da conexão de uma página. Ele verifica a validade do 
certificado HTTPS da origem principal e identifica a ocorrência de "conteúdo misto" (\textit{mixed content}), que ocorre quando 
uma página segura (HTTPS) carrega sub-recursos (como imagens ou scripts) de origens inseguras (HTTP) \cite{edge-security}. A ferramenta detalha o 
status do certificado e da conexão para cada origem \cite{edge-security}.

\subsection{Sensores}
\begin{figure}[!htb]
    \centering
    \caption{Ferramenta "Sensores" do Microsoft Edge Devtools}
    \includegraphics[width=0.8\linewidth]{assets/tools/edge-sensors.png}\\
    {\footnotesize Fonte: Microsoft Learn}
    \label{fig:edge-sensors}
\end{figure}
A ferramenta \textbf{Sensores} (Sensors), exibida na figura \ref{fig:edge-sensors}, é um utilitário de emulação de hardware e estados do sistema. Ela permite sobrescrever 
a geolocalização (latitude/longitude), simular a orientação física do dispositivo (dados de acelerômetro e giroscópio) e forçar 
eventos de toque \cite{edge-sensors}. Capacidades mais recentes incluem a emulação de estados do \textit{Idle Detector}, a simulação da concorrência 
de hardware (\texttt{navigator.hardwareConcurrency}) e a emulação da pressão da CPU \cite{edge-sensors}.

\subsection{Monitor de Mapas de Origem}
\begin{figure}[!htb]
    \centering
    \caption{Ferramenta "Monitor de Mapas de Origem" do Microsoft Edge Devtools}
    \includegraphics[width=0.8\linewidth]{assets/tools/edge-sourcemap-monitor.png}\\
    {\footnotesize Fonte: Microsoft Learn}
    \label{fig:edge-sourcemap-monitor}
\end{figure}
Este painel, apresentado na figura \ref{fig:edge-sourcemap-monitor}, monitora especificamente a requisição e o status de carregamento de mapas de origem (\textit{source maps}). Ele 
fornece um log de quais arquivos \textit{source map} foram solicitados, se foram carregados com sucesso ou se falharam \cite{edge-sourcemap-monitor}. É um 
utilitário de diagnóstico crucial para garantir que a depuração de código minificado ou transpilado (como TypeScript) funcione 
corretamente, mapeando o código executado de volta ao código-fonte original \cite{edge-sourcemap-monitor}.

\subsection{WebAuthn}
\begin{figure}[!htb]
    \centering
    \caption{Ferramenta "WebAuthn" do Microsoft Edge Devtools}
    \includegraphics[width=0.8\linewidth]{assets/tools/edge-webauthn.png}\\
    {\footnotesize Fonte: Microsoft Learn}
    \label{fig:edge-webauthn}
\end{figure}
A ferramenta \textbf{WebAuthn}, ilustrada na figura \ref{fig:edge-webauthn}, é um utilitário de depuração para a API de Autenticação da Web. Ela permite a criação e 
gerenciamento de autenticadores virtuais baseados em software, eliminando a necessidade de dispositivos físicos (como chaves de 
segurança USB) durante o desenvolvimento \cite{edge-webauthn}. O painel permite a configuração de atributos do autenticador, como protocolo (CTAP2, 
U2F), transporte e suporte a chaves residentes \cite{edge-webauthn}.

\subsection{WebAudio}
\begin{figure}[!htb]
    \centering
    \caption{Ferramenta "WebAudio" do Microsoft Edge Devtools}
    \includegraphics[width=0.8\linewidth]{assets/tools/edge-webaudio.png}\\
    {\footnotesize Fonte: Microsoft Learn}
    \label{fig:edge-webaudio}
\end{figure}
O painel \textbf{WebAudio}, apresentado na figura \ref{fig:edge-webaudio}, é um utilitário de diagnóstico para aplicações que utilizam a API WebAudio. Ele renderiza um grafo 
visual interativo de todos os nós de áudio (\textit{AudioNodes}) instanciados, mostrando suas conexões e estado \cite{edge-webaudio}. Esta 
visualização é essencial para depurar o fluxo de processamento de áudio, inspecionar os parâmetros dos nós e monitorar o 
\textit{AudioContext} em tempo real \cite{edge-webaudio}.