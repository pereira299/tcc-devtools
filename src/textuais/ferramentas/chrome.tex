\section{Google Chrome}
O navegador Google Chrome integra um conjunto de ferramentas de desenvolvimento web, conhecido como Chrome DevTools. Esta suíte é projetada para instrumentar, inspecionar, depurar e criar perfis de aplicações web diretamente no navegador. As ferramentas fornecem funcionalidades para uma variedade de tarefas, incluindo a manipulação do DOM, alteração de CSS, análise de desempenho de carregamento de página e monitoramento de solicitações de rede. A seção subsequente apresenta uma análise detalhada dos painéis que compõem este conjunto, elucidando as capacidades e finalidades específicas de cada ferramenta.

\subsection{Elementos}
\label{sec:chrome-elements}
\begin{figure}[!htb]
    \centering
    \caption{Ferramenta "Elementos" do Chrome Devtools}
    \includegraphics[width=0.65 \linewidth]{assets/tools/chrome-elements.png}\\
    {\footnotesize Fonte: Chrome for Developers}
    \label{fig:chrome-elements}
\end{figure}
A ferramenta ``Elementos'' (\textit{Elements}) é utilizada para realizar a inspeção e manipulação em 
tempo real do \textit{Document Object Model} (\textit{DOM}). Ele renderiza a estrutura da página em uma árvore 
\textit{DOM} hierárquica, como apresentado na Figura \ref{fig:chrome-elements}, permitindo a seleção precisa e a modificação direta de nós. As 
funcionalidades de edição permitem alterar dinamicamente o conteúdo textual, modificar atributos, alterar o tipo de nó e 
reordenar elementos estruturalmente \cite{chrome-elements-overview}. A ferramenta também facilita a depuração ao permitir 
forçar estados de elementos (como \textit{:hover} ou \textit{:focus}), ocultar ou excluir nós, e definir pontos 
de interrupção (\textit{breakpoints}) que pausam a execução do script mediante modificações específicas 
no \textit{DOM} \cite{chrome-elements-dom}.

Além da manipulação estrutural, esta ferramenta é central para a análise e depuração de \textit{CSS}. 
A sub-aba ``Estilos'' (\textit{Styles}) exibe o conjunto de regras \textit{CSS} aplicadas a um nó selecionado, 
detalhando a cascata e identificando propriedades \cite{chrome-elements-css,chrome-elements-css-issues}. De 
forma complementar, agrega ainda ferramentas especializadas para a depuração de \textit{layouts} 
complexos, como \textit{Grid} e \textit{Flexbox} \cite{chrome-elements-grid,chrome-elements-flexbox}.
\subsection{Console}
\label{sec:chrome-console}
\begin{figure}[!htb]
    \centering
    \caption{Ferramenta "Console" do Chrome Devtools}
    \includegraphics[width=0.65 \linewidth]{assets/tools/chrome-console.png}\\
    {\footnotesize Fonte: Chrome for Developers}
    \label{fig:chrome-console}
\end{figure}
O painel ``Console'' (\textit{Console}) atua como uma interface primária para o diagnóstico e interação 
com aplicações \textit{web}, como exibido na figura \ref{fig:chrome-console}, servindo a duas funções principais: a visualização de mensagens 
registradas e a execução de \textit{JavaScript} \cite{chrome-console-overview}. Desenvolvedores utilizam a \textit{API} do \textit{Console}, através de 
métodos, para enviar mensagens de diagnóstico durante a execução do código. Este mecanismo de 
\textit{logging} é fundamental para verificar a ordem de execução e inspecionar os valores das 
variáveis em pontos específicos, facilitando a depuração e a análise do comportamento do 
\textit{script}.

Adicionalmente, o \textit{Console} funciona como um ambiente \textit{REPL} (\textit{Read-Eval-Print Loop}), permitindo a 
execução interativa de código \textit{JavaScript} no contexto da página inspecionada \cite{chrome-console-overview}. Por possuir 
acesso total ao objeto `window` da página, os desenvolvedores podem utilizá-lo para interagir 
programaticamente com o \textit{DOM}, modificar dinamicamente o conteúdo da página ou testar novas 
funcionalidades isoladamente \cite{chrome-console-overview}.

\subsection{Problemas}
\label{sec:chrome-issues}
\begin{figure}[!htb]
    \centering
    \caption{Ferramenta "Problemas" do Chrome Devtools}
    \includegraphics[width=0. \linewidth]{assets/tools/chrome-issues.png}\\
    {\footnotesize Fonte: Chrome for Developers}
    \label{fig:chrome-issues}
\end{figure}
O painel ``Problemas'' (\textit{Issues}), apresentado na figura \ref{fig:chrome-issues}, é utilizado para diagnósticos e 
anomalias detectadas pelo navegador, como problemas de \textit{cookies}, conteúdo misto, erros de \textit{CORS} 
e violações da Política de Segurança de Conteúdo (\textit{CSP}) \cite{chrome-issues}. Sua função é reduzir a verbosidade e 
a desordem de notificações no ``Console'' \ref{sec:chrome-console}, apresentando os problemas de forma estruturada, 
agregada e acionável. Cada item reportado inclui uma descrição contextual, uma solução 
recomendada e uma seção de ``Recursos Afetados'' que fornece \textit{links} diretos para o contexto 
relevante em outras ferramentas, como ``Elementos'' \ref{sec:chrome-elements}. O recurso também permite agrupar 
os problemas por tipo e filtrar a exibição de problemas relacionados a \textit{cookies} de terceiros \cite{chrome-issues}.

\subsection{Lighthouse}
\label{sec:chrome-lighthouse}
\begin{figure}[!htb]
    \centering
    \caption{Ferramenta "Lighthouse" do Chrome Devtools}
    \includegraphics[width=0.65 \linewidth]{assets/tools/chrome-lighthouse.png}\\
    {\footnotesize Fonte: Chrome for Developers}
    \label{fig:chrome-lighthouse}
\end{figure}
Esta ferramenta é projetada para realizar auditorias abrangentes da qualidade de aplicações 
\textit{web}, como exibido na figura \ref{fig:chrome-lighthouse}. Sua principal função é gerar um relatório detalhado que avalia o \textit{site} em diversas 
categorias fundamentais, including Desempenho, Acessibilidade, Práticas 
Recomendadas e \textit{SEO} \cite{chrome-lighthouse}. A ferramenta permite a configuração do ambiente de teste, como a emulação de 
dispositivos móveis, a seleção de modos de análise e a simulação de condições de rede (limitação simulada) \cite{chrome-lighthouse}. A execução de uma 
auditoria estabelece um referencial quantitativo, essencial para medir o impacto de 
otimizações subsequentes, e fornece recomendações práticas sobre as mudanças que podem gerar maior 
impacto na \textit{performance}.

O relatório resultante da auditoria apresenta uma pontuação de desempenho geral e detalha métricas 
quantitativas cruciais, como \textit{FCP} (Primeira Exibição de Conteúdo) e \textit{TBT} (Tempo total de bloqueio) \cite{chrome-lighthouse-score}.
\subsection{Memória}
\label{sec:chrome-memory}
\begin{figure}[!htb]
    \centering
    \caption{Ferramenta "Memória" do Chrome Devtools}
    \includegraphics[width=0.65 \linewidth]{assets/tools/chrome-memory.png}\\
    {\footnotesize Fonte: Chrome for Developers}
    \label{fig:chrome-memory}
\end{figure}
O painel ``Memória'' (\textit{Memory}), apresentado na figura \ref{fig:chrome-memory}, fornece um conjunto de 
ferramentas de diagnóstico para a análise do uso da memória em aplicações \textit{web}. Ele é projetado para 
identificar problemas como vazamentos de memória (\textit{memory leaks}), inchaço de memória (\textit{memory bloat}) e 
coletas de lixo (\textit{garbage collections}) frequentes \cite{chrome-memory}. A ferramenta oferece quatro 
tipos de perfis de captura: ``Instantâneo de alocação heap'' (\textit{Heap snapshot}), que exibe a distribuição 
de memória entre objetos \textit{JavaScript} e nós \textit{DOM}; ``Instrumentação de alocação na linha do tempo'' (\textit{Allocation 
instrumentation on timeline}), que monitora alocações ao longo do tempo para isolar vazamentos; 
``Amostragem de alocação'' (\textit{Allocation sampling}), um método de baixo \textit{overhead} que registra alocações de 
memória por pilha de execução \textit{JavaScript}; e ``Elementos separados'' (\textit{Detached elements}), que identifica 
objetos retidos por referências \textit{JavaScript} \cite{chrome-memory}. A análise dos \textit{snapshots} de \textit{heap} permite 
a identificação de nós \textit{DOM} separados (\textit{detached DOM nodes}), uma causa comum de vazamentos de memória \cite{chrome-memory-snapshots}.
\newpage
\subsection{Rede}
\label{sec:chrome-network}
O painel ``Rede'' (\textit{Network}) é uma ferramenta de diagnóstico, como exibido na figura \ref{fig:chrome-network} abaixo, 
utilizada para monitorar e analisar a atividade de rede de uma aplicação \textit{web}, registrando todas as solicitações 
de recursos (\textit{requests}) e suas respectivas respostas (\textit{responses}) em um \textit{log} cronológico \cite{chrome-network-overview}. Sua finalidade 
primária é a verificação da transferência de recursos e a inspeção das propriedades de solicitações individuais. 
A seleção de um recurso específico permite uma análise detalhada através de sub-abas, que expõem os cabeçalhos 
\textit{HTTP} (\textit{Headers}), o conteúdo da resposta (\textit{Response}), a cadeia de causa da solicitação (\textit{Initiator}) e um detalhamento temporal da 
atividade de rede (\textit{Timing}) \cite{chrome-network}. A ferramenta também permite a exportação dos dados de atividade de rede em 
formato \textit{HAR} (\textit{HTTP Archive}) \cite{chrome-network}.
\begin{figure}[!htb]
    \centering
    \caption{Ferramenta "Rede" do Chrome Devtools}
    \includegraphics[width=0.65 \linewidth]{assets/tools/chrome-network.png}\\
    {\footnotesize Fonte: Chrome for Developers}
    \label{fig:chrome-network}
\end{figure}

\subsection{Desempenho}
\label{sec:chrome-performance}
O painel ``Desempenho'' (\textit{Performance}), exibido na figura \ref{fig:chrome-performance} abaixo, é uma 
ferramenta de diagnóstico avançada projetada para gravar e analisar perfis de desempenho da \textit{CPU} em 
aplicações \textit{web} \cite{chrome-performance}. Ele permite a captura detalhada da \textit{performance} 
tanto em tempo de execução (\textit{runtime}) quanto durante o carregamento da página (\textit{load}), com o objetivo de 
identificar gargalos e otimizar a utilização de recursos. A ferramenta monitora métricas vitais (\textit{Core} 
\textit{Web Vitals}) em tempo real, como \textit{LCP}, \textit{CLS} e \textit{INP}, e oferece a capacidade de comparar dados locais com os 
dados de campo agregados do \textit{Chrome UX Report} \cite{chrome-performance}. As configurações de captura permitem a simulação precisa 
das condições do usuário, notadamente através da limitação (\textit{throttling}) de \textit{CPU} e rede. O relatório de 
\textit{performance} resultante fornece uma visualização detalhada da atividade da \textit{thread} principal através de 
um gráfico de chamas (\textit{flame graph}), juntamente com métricas de memória, capturas de tela do processo de 
renderização e análises tabulares, como ``Bottom-Up'' e ``Árvore de Chamadas'' (\textit{Call Tree}) \cite{chrome-performance}.
\begin{figure}[!htb]
    \centering
    \caption{Ferramenta "Desempenho" do Chrome Devtools}
    \includegraphics[width=0.5 \linewidth]{assets/tools/chrome-performance.png}\\
    {\footnotesize Fonte: Chrome for Developers}
    \label{fig:chrome-performance}
\end{figure}

\subsection{Gravador}
\label{sec:chrome-recorder}
O painel "Gravador" (\textit{Recorder}), apresentado na figura \ref{fig:chrome-recorder} abaixo, constitui uma 
ferramenta de diagnóstico avançada, projetada para a análise e depuração de fluxos de interação do 
usuário em aplicações \textit{web} \cite{chrome-recorder}. Sua funcionalidade central 
permite a captura e reprodução de sequências de eventos do usuário, facilitando a identificação de 
anomalias em caminhos de conversão e a medição da \textit{performance}. A ferramenta oferece capacidades de 
edição interativa das etapas gravadas, permitindo a inserção de pontos de interrupção (\textit{breakpoints}) e 
a execução passo a passo para uma depuração detalhada \cite{chrome-recorder}. Adicionalmente, os fluxos de usuário capturados 
podem ser exportados em formatos estruturados, como \textit{JSON}, ou como \textit{scripts} de automação para bibliotecas 
externas, como o \textit{Puppeteer}, permitindo a integração com processos de teste automatizados \cite{chrome-recorder}.
\newpage
\begin{figure}[!htb]
    \centering
    \caption{Ferramenta "Gravador" do Chrome Devtools}
    \includegraphics[width=0.65 \linewidth]{assets/tools/chrome-recorder.png}\\
    {\footnotesize Fonte: Chrome for Developers}
    \label{fig:chrome-recorder}
\end{figure}
\subsection{Renderização}
\label{sec:chrome-rendering}
\begin{figure}[!htb]
    \centering
    \caption{Ferramenta "Renderização" do Chrome Devtools}
    \includegraphics[width=0.65 \linewidth]{assets/tools/chrome-render.png}\\
    {\footnotesize Fonte: Chrome for Developers}
    \label{fig:chrome-render}
\end{figure}
O painel "Renderização" (\textit{Rendering}), exibido na figura \ref{fig:chrome-render} é um conjunto de 
ferramentas de diagnóstico focado na análise de aspectos visuais e de \textit{performance} da renderização 
de conteúdo \textit{web}. Suas funcionalidades primárias permitem a identificação de problemas de desempenho, 
como repinturas (\textit{Paint Flashing}), mudanças de \textit{layout} (\textit{Layout Shifts}), problemas de rolagem, e a 
visualização de estatísticas de renderização e métricas das \textit{Core Web Vitals} \cite{chrome-render}. 
A ferramenta também oferece capacidades avançadas de emulação de recursos de mídia \textit{CSS}, permitindo 
testar como a página é renderizada sob diferentes condições, como esquemas de cores preferidos ou 
redução de movimento, sem necessidade de alteração no código-fonte \cite{chrome-render-emulate}. 
Adicionalmente, disponibiliza efeitos para depuração, incluindo a emulação de foco na página, a 
ativação de um modo escuro automático e a simulação de diversas deficiências visuais para análise de acessibilidade \cite{chrome-render-emulate}.

\subsection{Segurança}
\label{sec:chrome-security}
\begin{figure}[!htb]
    \centering
    \caption{Ferramenta "Segurança" do Chrome Devtools}
    \includegraphics[width=0.65 \linewidth]{assets/tools/chrome-security.png}\\
    {\footnotesize Fonte: Chrome for Developers}
    \label{fig:chrome-security}
\end{figure}
O painel "Privacidade e Segurança", apresentado acima na figura \ref{fig:chrome-security}, é uma ferramenta de 
diagnóstico que permite a inspeção e depuração de protocolos de segurança e o gerenciamento da privacidade da página. 
Esta funcionalidade possibilita a análise das origens dos recursos, exibindo detalhes da conexão, avisos de segurança 
\textit{HTTP} e informações de certificados \cite{chrome-security}. O painel identifica problemas críticos, como origens 
principais não seguras (solicitadas \textit{via} \textit{HTTP}), configurações de \textit{HTTPS} corrompido (por exemplo, certificados inválidos) 
e a ocorrência de conteúdo misto, onde recursos inseguros são requisitados em uma página segura. Adicionalmente, a 
seção de "Privacidade" oferece mecanismos para inspecionar e simular a limitação temporária de \textit{cookies} de terceiros, 
permitindo a verificação do comportamento do \textit{site} sob restrições de rastreamento \cite{chrome-security}.

\subsection{Fontes}
\label{sec:chrome-sources}
\begin{figure}[!htb]
    \centering
    \caption{Ferramenta "Fontes" do Chrome Devtools}
    \includegraphics[width=0.65 \linewidth]{assets/tools/chrome-source.png}\\
    {\footnotesize Fonte: Chrome for Developers}
    \label{fig:chrome-source}
\end{figure}
O painel "Fontes" (\textit{Sources}), como pode ser visto na figura \ref{fig:chrome-source}, constitui um ambiente 
integrado para a inspeção, edição e depuração dos recursos de uma aplicação \textit{web}. Ele permite o acesso e a visualização 
da árvore de arquivos carregados, incluindo \textit{scripts} \textit{JavaScript}, folhas de estilo \textit{CSS} e imagens. Sua funcionalidade principal 
é a depuração de \textit{JavaScript}, permitindo ao desenvolvedor definir pontos de interrupção (\textit{breakpoints}) para pausar intencionalmente a execução do código e 
analisar o fluxo de controle linha por linha \cite{chrome-source}. Adicionalmente, o painel funciona como um editor de código-fonte para 
\textit{CSS} e \textit{JavaScript}, possibilita a criação e execução de "\textit{Snippets}" (fragmentos de \textit{script} reutilizáveis) e suporta a 
configuração de "\textit{Workspaces}" para persistir as modificações feitas no navegador diretamente no sistema de arquivos 
local \cite{chrome-source}.

\subsection{Autofill}
\label{sec:chrome-autofill}
\begin{figure}[!htb]
    \centering
    \caption{Ferramenta "Autofill" do Chrome Devtools}
    \includegraphics[width=0.65 \linewidth]{assets/tools/chrome-autofill.png}\\
    {\footnotesize Fonte: Chrome for Developers}
    \label{fig:chrome-autofill}
\end{figure}
O painel "\textit{Autofill}" (Preenchimento automático), exibido na figura \ref{fig:chrome-autofill}, é uma ferramenta de 
diagnóstico utilizada para inspecionar e depurar o preenchimento de informações de endereço em formulários \textit{web}. 
Esta funcionalidade permite a análise do mapeamento entre os campos de formulário detectados, os valores previstos 
determinados heuristicamente pelo navegador e os dados de endereço reais salvos pelo usuário que são inseridos 
\cite{chrome-autofill}. O painel apresenta uma visualização tabular detalhada dessa correspondência e oferece a 
capacidade de injetar dados de endereço de teste para validar o comportamento do formulário \cite{chrome-autofill}. Adicionalmente, ele 
auxilia na identificação de problemas de implementação, que são relatados no painel "\textit{Issues}" (Problemas) 
\ref{sec:chrome-issues} para corrigir atributos de preenchimento automático incorretos.

\subsection{Animações}
\label{sec:chrome-animations}
\begin{figure}[!htb]
    \centering
    \caption{Ferramenta "Animações" do Chrome Devtools}
    \includegraphics[width=0.65 \linewidth]{assets/tools/chrome-animations.png}\\
    {\footnotesize Fonte: Chrome for Developers}
    \label{fig:chrome-animations}
\end{figure}
O painel "Animações" (\textit{Animations}), ilustrada na figura \ref{fig:chrome-animations}, é uma ferramenta de diagnóstico 
para a inspeção e modificação de sequências de animação. Ele captura e agrupa automaticamente animações \textit{CSS}, transições 
\textit{CSS}, animações da \textit{Web} (\textit{Web Animations API}) e a \textit{API} \textit{View Transitions}, embora não ofereça suporte a animações baseadas 
em \textit{requestAnimationFrame} \cite{chrome-animations}. A ferramenta permite a inspeção detalhada de grupos de animação, possibilitando ao desenvolvedor 
diminuir a velocidade, repetir ou analisar o código-fonte associado. As funcionalidades de modificação permitem o ajuste interativo de parâmetros 
temporais, como duração, atraso (\textit{delay}) e os deslocamentos de \textit{keyframes} \cite{chrome-animations}. Adicionalmente, o painel facilita a depuração 
da \textit{API} \textit{View Transitions}, permitindo a edição de seus pseudoelementos \textit{CSS} (ex: ::\textit{view-transition}) enquanto a execução da 
animação está pausada \cite{chrome-animations}.

\subsection{Aplicativo}
\label{sec:chrome-application}
\begin{figure}[!htb]
    \centering
    \caption{Ferramenta "Aplicativo" do Chrome Devtools}
    \includegraphics[width=0.65 \linewidth]{assets/tools/chrome-application.png}\\
    {\footnotesize Fonte: Chrome for Developers}
    \label{fig:chrome-application}
\end{figure}
O painel "\textit{Application}" (Aplicativo), exibida na figura \ref{fig:chrome-application}, é uma ferramenta de diagnóstico abrangente utilizada para inspecionar, modificar e 
depurar os diversos aspectos de armazenamento, manifesto e execução de uma aplicação \textit{web}. Ele permite a análise do 
\textit{manifest.json}, a inspeção e depuração de \textit{service workers}, including a emulação de eventos \textit{push}, e o gerenciamento de 
dados do \textit{site}, como a simulação de cotas de armazenamento \cite{chrome-application-pwa}. A ferramenta oferece acesso granular para visualização e edição 
de múltiplos mecanismos de armazenamento do lado do cliente, incluindo \textit{Local Storage}, \textit{Session Storage}, \textit{IndexedDB}, \textit{Cookies}, 
\textit{Shared Storage} e \textit{Cache Storage} \cite{chrome-application-storage}. Adicionalmente, o painel facilita o teste de serviços em segundo plano, como \textit{Background Fetch}, 
\textit{Background Sync} e \textit{Speculative Loads}, e a inspeção da estrutura de \textit{frames} da página e suas políticas de segurança associadas.

\subsection{Alterações}
\label{sec:chrome-changes}
\begin{figure}[!htb]
    \centering
    \caption{Ferramenta "Alterações" do Chrome Devtools}
    \includegraphics[width=0.65 \linewidth]{assets/tools/chrome-changes.png}\\
    {\footnotesize Fonte: Chrome for Developers}
    \label{fig:chrome-changes}
\end{figure}
O painel "Alterações" (\textit{Changes}), apresentado na figura \ref{fig:chrome-changes}, opera como uma ferramenta de monitoramento em tempo real para modificações de 
código-fonte executadas diretamente no ambiente de inspeção do navegador. Esta funcionalidade rastreia edições 
efetuadas em arquivos CSS, seja no painel "Elementos > Estilos" ou em "Fontes", e em arquivos JavaScript dentro 
da ferramenta "Fontes" \ref{sec:chrome-sources} \cite{chrome-changes}. O rastreamento de alterações em HTML também é suportado, contanto que o recurso "Substituições 
locais" (\textit{Local Overrides}) esteja habilitado. A ferramenta apresenta uma visualização \textit{diff} formatada automaticamente, 
que destaca inserções e exclusões de código \cite{chrome-changes}. Funcionalidades adicionais incluem a capacidade de copiar todas as alterações 
de CSS acumuladas e a opção de reverter todas as modificações aplicadas a um arquivo específico.

\subsection{Cobertura}
\label{sec:chrome-coverage}
\begin{figure}[!htb]
    \centering
    \caption{Ferramenta "Cobertura" do Chrome Devtools}
    \includegraphics[width=0.65 \linewidth]{assets/tools/chrome-coverage.png}\\
    {\footnotesize Fonte: Chrome for Developers}
    \label{fig:chrome-coverage}
\end{figure}
O painel "Cobertura" (\textit{Coverage}), exibido na figura \ref{fig:chrome-coverage}, é uma ferramenta de diagnóstico projetada para identificar código \textit{JavaScript} e \textit{CSS} 
não executado. Esta funcionalidade opera através da instrumentação do código durante uma sessão de gravação, que 
pode ser iniciada com um recarregamento da página, para monitorar quais partes do código são de fato utilizadas 
durante o carregamento e a interação do usuário \cite{chrome-coverage}. Ao concluir a gravação, a ferramenta gera um relatório que quantifica, 
para cada recurso analisado, o total de \textit{bytes} e o volume exato de "\textit{bytes} não usados". Adicionalmente, o painel "Cobertura" 
oferece uma visualização detalhada linha por linha na ferramenta "Fontes" \ref{sec:chrome-sources}, demarcando o código não utilizado. A análise pode ser 
configurada por escopo ("Por função" ou "Por bloco"), permitindo que desenvolvedores isolem e removam código supérfluo com o 
objetivo de otimizar o desempenho de carregamento da página e reduzir o consumo de dados da rede \cite{chrome-coverage}.

\subsection{Visão geral de CSS}
\label{sec:chrome-overview-css}
\begin{figure}[!htb]
    \centering
    \caption{Ferramenta "Visão geral de CSS" do Chrome Devtools}
    \includegraphics[width=0.65 \linewidth]{assets/tools/chrome-overview-css.png}\\
    {\footnotesize Fonte: Chrome for Developers}
    \label{fig:chrome-overview-css}
\end{figure}
O painel "Visão geral de \textit{CSS}" (\textit{CSS Overview}), como ilustrado na figura \ref{fig:chrome-overview-css}, é uma ferramenta de 
diagnóstico que, mediante uma captura iniciada pelo usuário, gera um relatório estatístico abrangente sobre a utilização 
de \textit{CSS} em uma página \textit{web}, com o objetivo de identificar potenciais otimizações \cite{chrome-overview-css}. Este relatório 
coleta dados de todas as ocorrências de \textit{CSS}, incluindo declarações não utilizadas, e estrutura a informação em seções 
principais: "Resumo da visão geral", "Cores", que também identifica problemas de baixo contraste, "Informações da fonte", 
"Declarações não utilizadas" e "\textit{Media queries}" \cite{chrome-overview-css}. A ferramenta permite a investigação interativa dos elementos afetados, 
oferecendo funcionalidades para destacá-los na página ou inspecioná-los diretamente no painel "Elementos".

\subsection{Recursos para desenvolvedores}
\label{sec:chrome-developer-resources}
\begin{figure}[!htb]
    \centering
    \caption{Ferramenta "Recursos para desenvolvedores" do Chrome Devtools}
    \includegraphics[width=0.65 \linewidth]{assets/tools/chrome-developer-resources.png}\\
    {\footnotesize Fonte: Chrome for Developers}
    \label{fig:chrome-developer-resources}
\end{figure}
O painel "Recursos para desenvolvedores" (\textit{Developer Resources}), exibida na figura \ref{fig:chrome-developer-resources},
é uma ferramenta de diagnóstico utilizada para verificar o status de carregamento dos mapas de origem (\textit{source maps}) pelo 
\textit{Chrome DevTools} \cite{chrome-developer-resources}. Por padrão, o \textit{DevTools} tenta carregar automaticamente os mapas de origem ao ser aberto, e este painel exibe o "\textit{Status}" e eventuais mensagens de "Erro" resultantes 
dessas tentativas. A ferramenta permite a filtragem dos recursos por \textit{URL} ou mensagem de erro e oferece opções para solucionar 
falhas de carregamento, como problemas de \textit{cross-origin}, ao permitir que os mapas sejam requisitados através do próprio \textit{site} \cite{chrome-developer-resources}. 
Adicionalmente, caso os mapas de origem não estejam presentes no ambiente, como em produção, o painel viabiliza o carregamento 
manual mediante o fornecimento de uma \textit{URL}, permitindo assim a depuração do código-fonte original.
\subsection{Camadas}
\label{sec:chrome-layers}
\begin{figure}[!htb]
    \centering
    \caption{Ferramenta "Camadas" do Chrome Devtools}
    \includegraphics[width=0.65 \linewidth]{assets/tools/chrome-layers.png}\\
    {\footnotesize Fonte: Chrome for Developers}
    \label{fig:chrome-layers}
\end{figure}
O painel "Camadas" (\textit{Layers}), apresentado na figura \ref{fig:chrome-layers}, é uma ferramenta de diagnóstico que permite 
a análise da composição de renderização de um \textit{website}. Ele apresenta um diagrama 3D interativo que ilustra como o 
navegador organiza o conteúdo em distintas camadas \cite{chrome-layers}. Esta visualização auxilia na identificação 
de problemas de renderização, na otimização de animações e na depuração de \textit{performance} de rolagem. A ferramenta lista 
todas as camadas renderizadas em uma árvore expansível e fornece detalhes como tamanho, contagem de pintura (\textit{Paint Count}) 
e motivos para a composição (\textit{Compositing Reasons}) \cite{chrome-layers}. Funcionalidades adicionais incluem a capacidade de inspecionar informações 
do "\textit{Paint Profiler}" e identificar regiões de rolagem lenta (\textit{Slow scroll rects}).

\subsection{Mídia}
\label{sec:chrome-midia}
\begin{figure}[!htb]
    \centering
    \caption{Ferramenta "Mídia" do Chrome Devtools}
    \includegraphics[width=0.65 \linewidth]{assets/tools/chrome-midia.png}\\
    {\footnotesize Fonte: Chrome for Developers}
    \label{fig:chrome-midia}
\end{figure}
O painel "Mídia" (\textit{Media}), ilustrado na figura \ref{fig:chrome-midia}, é a ferramenta primária para a inspeção e depuração de 
\textit{players} de mídia incorporados em uma página \textit{web}. Ele identifica e lista todas as fontes de áudio e vídeo ativas, permitindo a 
análise detalhada de cada \textit{player} \cite{chrome-midia}. A ferramenta é segmentada em abas que expõem dados técnicos: "\textit{Properties}" (Propriedades) exibe informações como resolução e \textit{codecs}; 
"\textit{Events}" (Eventos) registra todos os eventos emitidos pelo \textit{player}; "\textit{Messages}" (Mensagens) apresenta \textit{logs} de diagnóstico 
filtráveis; e "\textit{Timeline}" (Linha do tempo) visualiza em tempo real os estados de reprodução e \textit{buffer} \cite{chrome-midia}. O painel também permite 
a exportação das informações do \textit{player} para análise externa.

\subsection{Inspetor de memória}
\label{sec:chrome-memory-inspector}
\begin{figure}[!htb]
    \centering
    \caption{Ferramenta "Inspetor de memória" do Chrome Devtools}
    \includegraphics[width=0.65 \linewidth]{assets/tools/chrome-memory-inspector.jpg}\\
    {\footnotesize Fonte: Chrome for Developers}
    \label{fig:chrome-memory-inspector}
\end{figure}
O "Inspetor de memória" (\textit{Memory Inspector}), exibido na figura \ref{fig:chrome-memory-inspector}, é uma ferramenta de diagnóstico projetada para a inspeção de \textit{buffers} de memória 
binária, como \textit{ArrayBuffer}, \textit{TypedArray} e \textit{DataView} em \textit{JavaScript}, além da \textit{WebAssembly.Memory} de aplicações \textit{WebAssembly} (\textit{Wasm}) 
compiladas a partir de C++ \cite{chrome-memory-inspector}. A interface exibe o \textit{buffer} de memória apresentando os endereços de \textit{byte} e seus valores em formato 
hexadecimal, uma representação \textit{ASCII} adjacente e um "Inspetor de valor" que decodifica os \textit{bytes} selecionados em múltiplos 
formatos (por exemplo, ponto flutuante, inteiro) e codificações \cite{chrome-memory-inspector}. A ferramenta permite a navegação direta para endereços de memória 
específicos, a alternância de \textit{endianidade} (\textit{endianness}) e pode ser iniciada dinamicamente durante a depuração a partir do 
painel "Escopo" para analisar o estado da memória de um objeto em um ponto de interrupção \cite{chrome-memory-inspector}.

\subsection{Condições de rede}
\label{sec:chrome-network-conditions}
\begin{figure}[!htb]
    \centering
    \caption{Ferramenta "Condições de rede" do Chrome Devtools}
    \includegraphics[width=0.65 \linewidth]{assets/tools/chrome-network-conditions.png}\\
    {\footnotesize Fonte: Chrome for Developers}
    \label{fig:chrome-network-conditions}
\end{figure}
O painel "Condições de rede" (\textit{Network Conditions}), apresentado na figura \ref{fig:chrome-network-conditions}, é uma ferramenta de emulação que permite ao desenvolvedor substituir a 
\textit{string} do \textit{user agent} e simular diferentes velocidades de rede. A substituição da \textit{string} do \textit{user agent} altera a forma como 
o navegador se identifica para os servidores \textit{web}, o que é utilizado para testar \textit{design} responsivo, compatibilidade e detecção 
de recursos ao simular navegadores distintos ou versões anteriores \cite{chrome-network-conditions}. A ferramenta também permite a personalização das Dicas de 
Cliente \textit{HTTP} (\textit{User-Agent Client Hints}). A funcionalidade de limitação de rede (\textit{throttling}) possibilita a simulação de conexões 
de rede variadas, como 3G rápida, 3G lenta ou \textit{offline}, auxiliando na análise do comportamento da aplicação sob diferentes 
condições de conectividade \cite{chrome-network-conditions}.

\subsection{Bloqueio de solicitações de rede}
\label{sec:chrome-network-request-blocking}
\begin{figure}[!htb]
    \centering
    \caption{Ferramenta "Bloqueio de solicitações de rede" do Chrome Devtools}
    \includegraphics[width=0.65 \linewidth]{assets/tools/chrome-network-request-blocking.png}\\
    {\footnotesize Fonte: Chrome for Developers}
    \label{fig:chrome-network-request-blocking}
\end{figure}
O painel "Bloqueio de solicitações de rede" (\textit{Network Request Blocking}), exibido na figura \ref{fig:chrome-network-request-blocking}, 
é uma funcionalidade de diagnóstico utilizada para testar o comportamento de uma página \textit{web} sob a condição de falha no 
carregamento de recursos específicos, como imagens ou folhas de estilo \cite{chrome-network-request-blocking}. A ferramenta 
permite a definição de múltiplos "padrões" de bloqueio, os quais podem ser \textit{URLs} completos, \textit{URLs} parciais com correspondência 
de curinga (*), ou nomes de domínio \cite{chrome-network-request-blocking}. O desenvolvedor pode adicionar, editar, remover e 
alternar o estado (ativo/inativo) desses padrões. As regras de bloqueio também podem ser iniciadas contextualmente a partir 
do painel "Rede" \ref{sec:chrome-network}. Uma vez ativado, o painel contabiliza e exibe o número de solicitações interceptadas 
que correspondem a cada padrão definido \cite{chrome-network-request-blocking}.

\subsection{Monitor de Desempenho}
\label{sec:chrome-performance-monitor}
\begin{figure}[!htb]
    \centering
    \caption{Ferramenta "Monitor de Desempenho" do Chrome Devtools}
    \includegraphics[width=0.65 \linewidth]{assets/tools/chrome-performance-monitor.png}\\
    {\footnotesize Fonte: Chrome for Developers}
    \label{fig:chrome-performance-monitor}
\end{figure}
O "Monitor de Desempenho" (\textit{Performance Monitor}), como ilustrado na figura \ref{fig:chrome-performance-monitor}, é uma ferramenta 
de diagnóstico que fornece uma visualização em tempo real do desempenho de carregamento e execução de uma aplicação \textit{web}. A 
ferramenta apresenta uma linha do tempo que plota graficamente diversas métricas, permitindo que estas sejam ativadas ou 
desativadas para análise \cite{chrome-performance-monitor}. As métricas rastreadas incluem: uso da CPU, tamanho do \textit{heap} \textit{JavaScript}, 
o número total de nós \textit{DOM}, \textit{listeners} de eventos \textit{JavaScript}, documentos e \textit{frames}, bem como a frequência de \textit{layouts} e recálculos de estilo 
por segundo \cite{chrome-performance-monitor}. Além disso, o monitor persiste seus dados durante a navegação entre páginas, auxiliando na identificação de padrões de uso 
de recursos que podem indicar ineficiências de código ou problemas como \textit{layout thrashing}.

\subsection{Monitor de protocolo}
\label{sec:chrome-protocol-monitor}
\begin{figure}[!htb]
    \centering
    \caption{Ferramenta "Monitor de protocolo" do Chrome Devtools}
    \includegraphics[width=0.65 \linewidth]{assets/tools/chrome-protocol-monitor.png}\\
    {\footnotesize Fonte: Chrome for Developers}
    \label{fig:chrome-protocol-monitor}
\end{figure}
O "Monitor de protocolo" (\textit{Protocol Monitor}), apresentado na figura \ref{fig:chrome-protocol-monitor}, é uma ferramenta que permite 
a inspeção de todas as solicitações e respostas do Protocolo \textit{Chrome DevTools} (CDP). Esta funcionalidade registra as mensagens do CDP 
e permite a análise detalhada dos dados de solicitação e resposta \cite{chrome-protocol-monitor}. A ferramenta viabiliza o envio direto 
de comandos brutos do CDP, suportando tanto comandos simples sem parâmetros quanto comandos complexos com parâmetros estruturados em \textit{JSON}, 
auxiliado por um editor de comandos dedicado \cite{chrome-protocol-monitor}. Adicionalmente, o monitor permite o \textit{download} das mensagens 
registradas em formato \textit{JSON} para análise externa.

\subsection{Origem rápida}
\label{sec:chrome-quick-source}
\begin{figure}[!htb]
    \centering
    \caption{Ferramenta "Origem rápida" do Chrome Devtools}
    \includegraphics[width=0.65 \linewidth]{assets/tools/chrome-quick-source.png}\\
    {\footnotesize Fonte: Chrome for Developers}
    \label{fig:chrome-quick-source}
\end{figure}
O painel "Origem rápida" (\textit{Quick source}), exibido na figura \ref{fig:chrome-quick-source}, opera como uma interface suplementar 
para visualização e edição de arquivos de origem. Sua principal utilidade reside na sua integração na "gaveta" (\textit{drawer}), por 
padrão na parte inferior da janela do \textit{DevTools}, o que permite ao desenvolvedor inspecionar e modificar o código-fonte enquanto 
mantém acesso simultâneo a outros painéis \cite{chrome-quick-source}. Esta funcionalidade elimina a necessidade de alternar repetidamente entre a ferramenta 
"Fontes" (\textit{Sources}) \ref{sec:chrome-sources} principal e outras ferramentas. O painel "Origem rápida" abre automaticamente o último 
arquivo editado na ferramenta "Fontes" e permite a abertura de outros arquivos através de atalhos de teclado (\textit{Command+P} ou \textit{Ctrl+P}), 
que são contextualmente redirecionados para este painel quando ele está em foco \cite{chrome-quick-source}.

\subsection{WebAudio}
\label{sec:chrome-webaudio}
\begin{figure}[!htb]
    \centering
    \caption{Ferramenta "WebAudio" do Chrome Devtools}
    \includegraphics[width=0.65 \linewidth]{assets/tools/chrome-webaudio.png}\\
    {\footnotesize Fonte: Chrome for Developers}
    \label{fig:chrome-webaudio}
\end{figure}
O painel "\textit{WebAudio}", apresentado na figura \ref{fig:chrome-webaudio}, é uma ferramenta de diagnóstico que exibe métricas de 
desempenho para instâncias de \textit{AudioContext} em aplicações que utilizam a API \textit{WebAudio} \cite{chrome-webaudio}. Após a seleção de um contexto de áudio 
específico, o painel apresenta métricas chave, incluindo o Estado (\textit{State}) operacional (por exemplo, \textit{running}, \textit{suspended}), a 
Taxa de Amostragem (\textit{Sample Rate}) em Hz, o Tamanho do \textit{Buffer} de \textit{Callback} (\textit{Callback Buffer Size}) em quadros e o Número Máximo de 
Canais de Saída (\textit{Max Output Channel Count}) \cite{chrome-webaudio}. Adicionalmente, a ferramenta monitora em tempo real o Intervalo de \textit{Callback} (\textit{Callback Interval}) 
e a Capacidade de Renderização (\textit{Render Capacity}) do processador de áudio.
\subsection{Sensores}
\begin{figure}[!htb]
    \centering
    \caption{Ferramenta "Sensores" do Chrome Devtools}
    \includegraphics[width=0.65 \linewidth]{assets/tools/chrome-sensors.png}\\
    {\footnotesize Fonte: Chrome for Developers}
    \label{fig:chrome-sensors}
\end{figure}
O painel "Sensores" (\textit{Sensors}), ilustrado na figura \ref{fig:chrome-sensors}, é uma ferramenta de emulação projetada para simular 
diversas entradas de \textit{hardware} e estados do dispositivo. Suas principais funcionalidades incluem a substituição de dados de 
geolocalização, a simulação de orientação do dispositivo, a forçagem de eventos de toque e a emulação de estados da API \textit{Idle Detection} 
(detector inativo) \cite{chrome-sensors}. Adicionalmente, a ferramenta viabiliza a substituição do valor de simultaneidade de \textit{hardware} 
(\textit{navigator.hardwareConcurrency}) e a simulação de estados da API \textit{Compute Pressure} (pressão da \textit{CPU}).


\subsection{WebAuthn} 
\label{sec:chrome-webauthn}
\begin{figure}[!htb]
    \centering
    \caption{Ferramenta "WebAuthn" do Chrome Devtools}
    \includegraphics[width=0.65 \linewidth]{assets/tools/chrome-webauthn.png}\\
    {\footnotesize Fonte: Chrome for Developers}
    \label{fig:chrome-webauthn}
\end{figure}
O painel "WebAuthn", apresentado na figura \ref{fig:chrome-webauthn}, fornece uma interface para a criação e interação com autenticadores virtuais baseados em software, 
permitindo a depuração da API Web Authentication \cite{chrome-webauthn}. A ferramenta possibilita a ativação de um ambiente de autenticador virtual, 
no qual desenvolvedores podem adicionar, renomear e remover autenticadores. É possível configurar o protocolo (por exemplo, ctap2, u2f)
e o transporte (por exemplo, USB, NFC) de cada autenticador, além de registrar credenciais e monitorar seus IDs e contagens de 
assinaturas durante os testes \cite{chrome-webauthn}.

