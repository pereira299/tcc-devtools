\section{Mozilla Firefox}
O navegador Mozilla Firefox integra um conjunto abrangente de ferramentas de desenvolvimento web, centralizadas na interface denominada "Toolbox" (Caixa de Ferramentas). Esta suíte é projetada para permitir que desenvolvedores inspecionem, depurem, monitorem e modifiquem aplicações web diretamente no navegador. A "Toolbox" agrupa a maioria das ferramentas de desenvolvedor incorporadas e pode ser acessada através do menu de contexto (Inspecionar), do menu principal do navegador ou por atalhos de teclado. As subseções seguintes detalham os componentes individuais desta suíte de ferramentas.

\subsection{Inspetor}
\begin{figure}[!htb]
    \centering
    \caption{Ferramenta "Inspetor" do Firefox Devtools}
    \includegraphics[width=0.65 \linewidth]{assets/tools/firefox-inspector.png}\\
    {\footnotesize Fonte: Firefox Source Tree Documentation}
    \label{fig:firefox-inspector}
\end{figure}
O Inspetor de Página (Page Inspector), como apresentado na figura \ref{fig:firefox-inspector}, é um componente central das 
ferramentas de desenvolvedor do Firefox, concebido para a inspeção e modificação em tempo real da estrutura HTML e das 
folhas de estilo (CSS) de um documento web \cite{firefox-inspector}. A ferramenta permite a análise e manipulação direta do Document Object Model (DOM), 
a edição de propriedades CSS, a avaliação do modelo de caixa (box model), e a depuração de layouts complexos, incluindo 
CSS Grid e Flexbox \cite{firefox-inspector}. Suas funcionalidades estendem-se à inspeção de manipuladores de eventos (event listeners), análise de animações, edição de fontes e seleção de cores, operando tanto em 
instâncias locais do navegador quanto em alvos de depuração remotos.

\subsection{Console}
\begin{figure}[!htb]
    \centering
    \caption{Ferramenta "Console" do Firefox Devtools}
    \includegraphics[width=0.65 \linewidth]{assets/tools/firefox-console.png}\\
    {\footnotesize Fonte: Firefox Source Tree Documentation}
    \label{fig:firefox-console}
\end{figure}
O Console Web (Web Console), exibido na figura \ref{fig:firefox-console}, é um componente das ferramentas de desenvolvedor do Firefox que opera como uma interface 
interativa para registro e depuração. Sua principal função é registrar informações detalhadas associadas a uma página web, 
incluindo requisições de rede, erros e advertências de JavaScript, CSS e segurança, além de mensagens de erro, aviso e 
informacionais explicitamente registradas pelo código JavaScript executado no contexto da página \cite{firefox-console}. Adicionalmente, a ferramenta 
viabiliza a interação direta com o documento, permitindo a execução de expressões JavaScript no contexto da página para fins 
de análise e manipulação dinâmica.

\subsection{Depurador}
\begin{figure}[!htb]
    \centering
    \caption{Ferramenta "Depurador" do Firefox Devtools}
    \includegraphics[width=0.65 \linewidth]{assets/tools/firefox-debugger.png}\\
    {\footnotesize Fonte: Firefox Source Tree Documentation}
    \label{fig:firefox-debugger}
\end{figure}
O Depurador JavaScript (JavaScript Debugger), como ilustrado na figura \ref{fig:firefox-debugger}, é uma ferramenta de 
diagnóstico que permite a análise e retificação de código JavaScript, possibilitando a execução passo a passo (step-through) 
e a inspeção ou modificação do estado do script para facilitar a identificação de bugs \cite{firefox-debugger}. A ferramenta é capaz de depurar código 
executado localmente no Firefox ou em alvos remotos, como o Firefox for Android. Suas funcionalidades centrais incluem a habilidade 
de pausar a execução em pontos específicos através de breakpoints, breakpoints condicionais, pontos de interrupção em XHR, listeners de eventos, exceções ou mutações do 
DOM \cite{firefox-debugger}. Adicionalmente, oferece controle sobre o fluxo de execução, inspeção de valores, suporte a source maps, formatação de 
código minificado (pretty-printing) e depuração de worker threads.
\subsection{Rede}
\begin{figure}[!htb]
    \centering
    \caption{Ferramenta "Rede" do Firefox Devtools}
    \includegraphics[width=0.65 \linewidth]{assets/tools/firefox-network.png}\\
    {\footnotesize Fonte: Firefox Source Tree Documentation}
    \label{fig:firefox-network}
\end{figure}
O Monitor de Rede (Network Monitor), exibido na figura \ref{fig:firefox-network}, é a ferramenta designada para exibir todas 
as requisições HTTP que o Firefox executa, incluindo carregamentos de página e XMLHttpRequests. Esta funcionalidade permite 
a análise da duração e dos detalhes de cada transação. A ferramenta registra a atividade de rede continuamente enquanto a "Toolbox" 
(Caixa de Ferramentas) estiver aberta, independentemente de o painel de Rede estar ativamente selecionado \cite{firefox-network}. 
Suas capacidades estendem-se à inspeção de Web Sockets, Server-Sent Events (SSE) e à simulação de diferentes condições de rede 
(throttling) \cite{firefox-network}.

\subsection{Editor de Estilo}
O Editor de Estilo (Style Editor) é um componente das ferramentas de desenvolvedor do Firefox que faculta a inspeção e 
manipulação das folhas de estilo (stylesheets) associadas a um documento web. A ferramenta permite a edição em tempo real do
CSS existente, com alterações aplicadas imediatamente à página, bem como a criação de novos stylesheets ou a importação de 
folhas de estilo externas \cite{firefox-style-editor}. Sua interface inclui um painel de editor de código e uma barra lateral dedicada à navegação de 
At-rules, como @media, @supports, @layer e @container. Além disso, o Editor de Estilo oferece suporte a "source maps" 
(mapas de origem), permitindo que desenvolvedores depurem o código-fonte original (ex: Sass, Less) em vez do CSS transpilado ou 
minificado \cite{firefox-style-editor}.

\subsection{Desempenho}
\begin{figure}[!htb]
    \centering
    \caption{Ferramenta "Desempenho" do Firefox Devtools}
    \includegraphics[width=0.65 \linewidth]{assets/tools/firefox-performance.png}\\
    {\footnotesize Fonte: Firefox Source Tree Documentation}
    \label{fig:firefox-performance}
\end{figure}
O painel Desempenho (Performance Panel), como apresentado na figura \ref{fig:firefox-performance}, é a ferramenta designada 
para a análise da responsividade geral, da execução de JavaScript e do desempenho de layout de uma aplicação web \cite{firefox-performance}. 
Sua operação baseia-se na captura de um perfil de desempenho, que é subsequentemente carregado na interface de usuário do Firefox Profiler, 
uma aplicação web externa, para inspeção detalhada \cite{firefox-performance}. Os dados capturados permanecem armazenados 
localmente, sendo que a ferramenta facilita o upload e compartilhamento desses perfis para análise colaborativa.

\subsection{Inspetor de acessibilidade}
\begin{figure}[!htb]
    \centering
    \caption{Ferramenta "Inspetor de Acessibilidade" do Firefox Devtools}
    \includegraphics[width=0.65 \linewidth]{assets/tools/firefox-accessiblity.png}\\
    {\footnotesize Fonte: Firefox Source Tree Documentation}
    \label{fig:firefox-accessibility}
\end{figure}
O Inspetor de Acessibilidade (Accessibility Inspector), exibido na figura \ref{fig:firefox-accessibility}, é um componente 
das ferramentas de desenvolvedor do Firefox que fornece acesso à árvore de acessibilidade (accessibility tree) da página 
\cite{firefox-accessibility}. Sua finalidade é permitir a inspeção da estrutura semântica exposta às tecnologias assistivas, 
possibilitando a verificação de atributos, a identificação de elementos ausentes ou a avaliação de componentes que necessitem 
de atenção para garantir a conformidade \cite{firefox-accessibility}.

\subsection{Aplicações}
\begin{figure}[!htb]
    \centering
    \caption{Ferramenta "Aplicações" do Firefox Devtools}
    \includegraphics[width=0.65 \linewidth]{assets/tools/firefox-application.png}\\
    {\footnotesize Fonte: Firefox Source Tree Documentation}
    \label{fig:firefox-application}
\end{figure}
O painel Aplicação (Application panel), apresentado na figura \ref{fig:firefox-application}, é um componente das ferramentas de desenvolvedor do Firefox que fornece instrumentos 
para a inspeção e depuração de aplicações web modernas, comumente designadas como Progressive Web Apps (PWAs) \cite{firefox-application}. Suas capacidades 
incluem especificamente a inspeção de service workers e de manifestos de aplicações web (web app manifests) \cite{firefox-application}.

\subsection{Memória}
\begin{figure}[!htb]
    \centering
    \caption{Ferramenta "Memória" do Firefox Devtools}
    \includegraphics[width=0.65 \linewidth]{assets/tools/firefox-memory.png}\\
    {\footnotesize Fonte: Firefox Source Tree Documentation}
    \label{fig:firefox-memory}
\end{figure}
A ferramenta Memória (Memory tool), exibida na figura \ref{fig:firefox-memory}, destina-se à análise do uso de memória, permitindo a captura de um instantâneo (snapshot) 
do heap de memória do separador atual. Esta funcionalidade fornece múltiplas visualizações do heap, que demonstram quais 
objetos são responsáveis pelo consumo de memória e os locais exatos no código onde as alocações estão ocorrendo. As 
visualizações disponíveis incluem "Tree map" (mapa de árvore), "Aggregate" (agregada) — que apresenta uma tabela de tipos 
alocados — e "Dominators" (dominadores) \cite{firefox-memory}. A visão "Dominators" é notável por exibir o "tamanho retido" (retained size) dos 
objetos, definido como o tamanho do próprio objeto acrescido do tamanho de outros objetos que ele mantém vivos através de 
referências \cite{firefox-memory}. Adicionalmente, a ferramenta suporta a comparação entre múltiplos instantâneos e o registro de pilhas de chamadas 
(call stacks) para rastrear a origem das alocações.

\subsection{Inspetor de armazenamento}
\begin{figure}[!htb]
    \centering
    \caption{Ferramenta "Inspetor de Armazenamento" do Firefox Devtools}
    \includegraphics[width=0.65 \linewidth]{assets/tools/firefox-storage.png}\\
    {\footnotesize Fonte: Firefox Source Tree Documentation}
    \label{fig:firefox-storage}
\end{figure}
O Inspetor de Armazenamento (Storage Inspector), ilustrado na figura \ref{fig:firefox-storage}, é a ferramenta designada para a inspeção dos diversos tipos de armazenamento 
que uma página web pode utilizar. Atualmente, o inspetor suporta a análise de Cache Storage (caches DOM criados através da 
Cache API), Cookies (criados pela página ou por iframes nela contidos), bancos de dados IndexedDB (incluindo seus Object Stores 
e os itens neles armazenados), Local Storage e Session Storage \cite{firefox-storage}. A interface organiza os objetos de armazenamento hierarquicamente 
por origem e, no presente momento, fornece uma visualização estritamente de leitura (read-only) dos dados armazenados \cite{firefox-storage}.

\subsection{Visualizador de Propriedades DOM}
\begin{figure}[!htb]
    \centering
    \caption{Ferramenta "Visualizador de Propriedades DOM" do Firefox Devtools}
    \includegraphics[width=0.65 \linewidth]{assets/tools/firefox-dom.png}\\
    {\footnotesize Fonte: Firefox Source Tree Documentation}
    \label{fig:firefox-dom}
\end{figure}
O Visualizador de Propriedades DOM (DOM Property Viewer), apresentado na figura \ref{fig:firefox-dom}, é uma ferramenta de inspeção que faculta a análise das propriedades 
do Document Object Model (DOM). A ferramenta apresenta essas propriedades em uma estrutura de árvore expansível, iniciando a 
inspeção a partir do objeto window global da página corrente ou de um iframe selecionado \cite{firefox-dom}. A interface exibe o nome de cada 
propriedade e seu valor correspondente, indicando visualmente propriedades não graváveis (non-writable) e permitindo a filtragem 
da lista para localizar itens específicos \cite{firefox-dom}. A exibição pode ser atualizada manualmente para refletir alterações no DOM.

\subsection{Conta-gotas}
\begin{figure}[!htb]
    \centering
    \caption{Ferramenta "Conta-gotas" do Firefox Devtools}
    \includegraphics[width=0.65 \linewidth]{assets/tools/firefox-eyedropper.png}\\
    {\footnotesize Fonte: Firefox Source Tree Documentation}
    \label{fig:firefox-eyedropper}
\end{figure}
O Conta-gotas (Eyedropper), exibido na figura \ref{fig:firefox-eyedropper}, é uma ferramenta que permite a seleção de cores da página web atual, operando como uma lupa para 
seleção com precisão de pixel e exibindo o valor de cor do pixel corrente. A ferramenta possui duas utilizações principais: 
selecionar uma cor da página para copiá-la para a área de transferência, ou modificar um valor de cor diretamente na 
visualização de Regras (Rules view) do Inspetor \cite{firefox-eyedropper}. Esta segunda funcionalidade é acessada através do seletor de cores (color 
picker) da visualização de Regras, onde a ativação do ícone do conta-gotas permite que a cor selecionada na página atualize a 
propriedade CSS correspondente \cite{firefox-eyedropper}.

\subsection{Captura de Tela}
\begin{figure}[!htb]
    \centering
    \caption{Ferramenta "Captura de Tela" do Firefox Devtools}
    \includegraphics[width=0.65 \linewidth]{assets/tools/firefox-screenshot.png}\\
    {\footnotesize Fonte: Firefox Source Tree Documentation}
    \label{fig:firefox-screenshot}
\end{figure}
A funcionalidade de Captura de Tela (Screenshot), apresentada na figura \ref{fig:firefox-screenshot}, permite a geração de imagens da página web renderizada, abrangendo tanto a 
captura da página inteira (full-page screenshot) quanto a de um nó (node) específico do DOM. A ferramenta é invocada através de 
um ícone opcional na barra de ferramentas, pelo menu de contexto de um elemento no Inspetor de HTML, ou programaticamente pelo 
Console Web \cite{firefox-screenshot}. A partir da versão 62 do Firefox, a função auxiliar :screenshot do console faculta controle granular, permitindo a definição 
de atrasos (delay), a especificação de um seletor CSS para o alvo, a alteração da razão de pixels do dispositivo (DPR) e a 
nomeação do arquivo de saída. Por padrão, as imagens são salvas no diretório de "Downloads", podendo alternativamente ser 
copiadas para a área de transferência \cite{firefox-screenshot}.

\subsection{Régua}
\begin{figure}[!htb]
    \centering
    \caption{Ferramenta "Régua" do Firefox Devtools}
    \includegraphics[width=0.65 \linewidth]{assets/tools/firefox-rulers.png}\\
    {\footnotesize Fonte: Firefox Source Tree Documentation}
    \label{fig:firefox-rulers}
\end{figure}
A ferramenta Réguas (Rulers), exibida na figura \ref{fig:firefox-rulers}, proporciona a sobreposição de guias dimensionais, especificamente réguas horizontais e verticais, 
diretamente sobre a página web renderizada \cite{firefox-rulers}. As unidades de medida utilizadas são pixels. Adicionalmente, a ferramenta exibe as 
dimensões atuais do viewport (área de visualização) próximo ao canto superior direito. Esta funcionalidade não é persistente, 
exigindo que o comando seja reativado após cada atualização da página ou navegação \cite{firefox-rulers}.

\subsection{Formatadores personalizados}
Os Formatadores Personalizados (Custom Formatters), ilustrado na figura \ref{fig:firefox-custom-formatters}, são uma funcionalidade que permite a um website controlar a renderização de 
variáveis JavaScript dentro do Console Web e do Depurador, visando aprimorar o processo de depuração ao fornecer uma 
representação mais intuitiva de objetos complexos \cite{firefox-custom-formatters}. A implementação é realizada através da definição de um array global 
denominado devtoolsFormatters, cujos elementos são objetos formatadores. Cada formatador deve conter uma função header e, 
opcionalmente, funções hasBody e body \cite{firefox-custom-formatters}. Essas funções retornam null ou um array baseado no padrão JsonML para construir a 
interface HTML customizada, suportando a referenciação de objetos aninhados através de um template de objeto específico.

\subsection{Traçador JavaScript}
\begin{figure}[!htb]
    \centering
    \caption{Ferramenta "Traçador JavaScript" do Firefox Devtools}
    \includegraphics[width=0.65 \linewidth]{assets/tools/firefox-tracer.png}\\
    {\footnotesize Fonte: Firefox Source Tree Documentation}
    \label{fig:firefox-tracer}
\end{figure}
O Traçador JavaScript (JavaScript Tracer), exibido na figura \ref{fig:firefox-tracer}, é uma ferramenta experimental do Firefox, ativada através da preferência 
devtools.debugger.features.javascript-tracing, concebida para registrar todas as chamadas de função JavaScript. A ferramenta 
permite a configuração de múltiplos destinos de saída (logging output), incluindo o Console Web, uma barra lateral no Depurador, 
um registro no Firefox Profiler ou a saída padrão (stdout) \cite{firefox-tracer}. Suas opções permitem um início de rastreamento atrasado (delayed 
start), condicionado à interação do usuário ou ao carregamento da página, e pode opcionalmente capturar retornos de função, 
valores de argumentos, mutações do DOM, bem como operar com limites de profundidade de pilha (depth limit) ou de número de 
registros (record limit) \cite{firefox-tracer}.