\section{Justificativa}
\label{sec:justificativa}

A crescente demanda por soluções digitais e a evolução constante da internet impulsionam o desenvolvimento de aplicações web cada vez mais sofisticadas e complexas. Nesse contexto, as ferramentas de desenvolvimento web desempenham um papel fundamental, permitindo que os desenvolvedores criem experiências online inovadoras e eficientes. No entanto, a diversidade de navegadores disponíveis no mercado e a constante atualização de suas funcionalidades tornam a escolha da ferramenta ideal e a garantia de compatibilidade tarefas desafiadoras.

Essa complexidade não é apenas uma questão técnica, pois a relevância do desenvolvimento web transcende os limites da área tecnológica, impactando diretamente a sociedade na totalidade. A internet se tornou uma plataforma indispensável para comunicação, comércio, educação e entretenimento, e as aplicações web são os alicerces que sustentam essa infraestrutura digital \cite{tecnologia-web}. Portanto, a compreensão das ferramentas e técnicas utilizadas nesse processo é crucial para acompanhar a evolução tecnológica e atender às demandas de um mercado cada vez mais exigente.

Diante desse cenário, este trabalho delimita seu escopo de análise técnica aos quatro navegadores de maior relevância mercadológica e estrutural na atualidade: Google Chrome, Microsoft Edge, Apple Safari e Mozilla Firefox. A escolha deste conjunto justifica-se pela representatividade de mercado, visto que, somados, estes softwares concentram a vasta maioria do tráfego web global, garantindo o alcance demográfico das aplicações. Além disso, sob a ótica da Engenharia de Software, este grupo abrange as três principais famílias de motores de renderização (Rendering Engines) vigentes — Blink, WebKit e Gecko —, tornando-os indispensáveis para a validação de interoperabilidade (cross-browser testing) e aderência aos padrões web (W3C). Assim, ao dissecar as especificidades de cada ferramenta, o estudo visa auxiliar desenvolvedores a tomar decisões fundamentadas, impulsionando a qualidade final dos projetos web.