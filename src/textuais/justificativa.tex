\chapter{Justificativa}
\label{chap:justificativa}

A crescente demanda por soluções digitais e a evolução constante da internet impulsionam o desenvolvimento de aplicações web cada vez mais sofisticadas e complexas. Nesse contexto, as ferramentas de desenvolvimento web desempenham um papel fundamental, permitindo que os desenvolvedores criem experiências online inovadoras e eficientes. No entanto, a diversidade de navegadores disponíveis no mercado e a constante atualização de suas funcionalidades tornam a escolha da ferramenta ideal uma tarefa desafiadora.

\section{Relevância do desenvolvimento web}
A relevância do desenvolvimento web transcende os limites da área tecnológica, impactando diretamente a sociedade na totalidade. A internet se tornou uma plataforma indispensável para comunicação, comércio, educação e entretenimento, e as aplicações web são os alicerces que sustentam essa infraestrutura digital \cite{tecnologia-web}. A compreensão das ferramentas e técnicas utilizadas nesse processo é crucial para acompanhar a evolução tecnológica e atender às demandas de um mercado cada vez mais exigente.

\section{Lacunas de conhecimento}
Apesar da vasta quantidade de informações disponíveis sobre desenvolvimento web, ainda existem lacunas de conhecimento em relação à comparação detalhada entre os principais navegadores e suas aplicações práticas. Dentre essas dificuldades, destaca-se a compreensão limitada sobre as ferramentas de análise de código estático, que possibilitam a identificação de erros antes da execução do código, contribuindo para a detecção precoce de inconsistências e a melhoria da qualidade do software \cite{odell_2014_pro}. Essa lacuna de conhecimento pode levar à escolha de ferramentas subótimas, impactando negativamente a qualidade, o desempenho e a manutenibilidade das aplicações.

\section{Contribuição com a comunidade}
Este trabalho visa contribuir para a comunidade de desenvolvedores web ao fornecer uma análise exploratória dos principais navegadores disponíveis no mercado. Ao identificar as vantagens e desvantagens de cada ferramenta, o estudo poderá auxiliar os desenvolvedores a tomar decisões mais informadas na escolha da ferramenta ideal para seus projetos. Além disso, a pesquisa poderá identificar tendências e direcionar futuras pesquisas na área de desenvolvimento web, impulsionando a evolução da área.
