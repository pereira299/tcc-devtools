\chapter{Pesquisa de campo}
Entre janeiro e abril de 2025, foi conduzida uma pesquisa de campo para compreender a percepção dos desenvolvedores de software sobre as \textit{devtools}, explorando suas preferências, usos no cotidiano e impactos em diferentes contextos e níveis de experiência. O objetivo foi identificar as funcionalidades mais valorizadas, os desafios enfrentados e a eficiência dessas ferramentas no fluxo de trabalho dos profissionais.

A pesquisa utilizou um formulário estruturado, aplicado por meio de uma plataforma digital, para coletar dados de um público amplo e diversificado. O formulário incluiu questões específicas para filtrar respostas, garantindo que apenas desenvolvedores fossem considerados, excluindo participantes fora do perfil, como não desenvolvedores web, o que assegurou a confiabilidade e validade dos resultados.

Ao todo, foram obtidas um total de 80 respostas, com participantes de diferentes estados brasileiros e diferentes níveis de experiência. Após a aplicação dos critérios de seleção, que filtraram os participantes para incluir apenas desenvolvedores web, 45 respostas foram selecionadas para análise, garantindo a relevância e a confiabilidade dos dados utilizados no estudo. 

\section{Resultados obtidos}
Os resultados revelam que as ferramentas de desenvolvimento são essenciais para o desenvolvimento front-end, sendo o Google Chrome o navegador dominante e a inspeção de elementos e o console os recursos mais utilizados. Embora muitos desenvolvedores considerem as DevTools suficientes, especialmente em níveis júnior e intermediário, desenvolvedores seniores frequentemente as complementam com ferramentas ou plugins de IDE, indicando que a complexidade do projeto influencia as preferências de ferramentas. 

Um fato interessante é que apenas 27\% dos desenvolvedores sabiam que o \textit{Chrome DevTools} oferece mais de 20 ferramentas, apesar de 67\% usarem o \textit{DevTools} diariamente. Isso sugere uma lacuna no conhecimento de todos os recursos oferecidos, especialmente entre desenvolvedores juniores e intermediários, que frequentemente utilizam um conjunto limitado de ferramentas, como inspeção de elementos e console.

A dificuldade moderada percebida por desenvolvedores juniores e sua dependência de recursos externos, como tutoriais e colegas, ressaltam a necessidade de melhores recursos educacionais para preencher a lacuna de conhecimento. Essas percepções sugerem que, embora essas ferramentas sejam valiosas, o aumento da divulgação e a oferta de guias e treinamento sobre seus recursos completos podem aprimorar sua eficácia no desenvolvimento web.