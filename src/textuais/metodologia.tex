\section{Metodologia}
\label{chap:metodologia}

Para o desenvolvimento deste trabalho, adotou-se a metodologia de \textbf{Pesquisa Documental}, fundamentada na análise rigorosa de \textbf{Documentação Técnica}. Esta abordagem é imperativa para o tema, visto que as ferramentas de desenvolvimento web (DevTools) são regidas por especificações técnicas complexas, documentação oficial fornecida pelos fabricantes (como Google, Mozilla, Microsoft e Apple) e padrões da web (W3C). Nesse cenário, a pesquisa documental permite extrair dados fidedignos e objetivos diretamente das fontes primárias, garantindo uma comparação baseada em fatos técnicos e não apenas em observações empíricas.

Teoricamente, a pesquisa documental distingue-se pelo uso de fontes que ainda não receberam tratamento analítico, servindo como base para investigações onde se busca precisão técnica e histórica \cite{gil-metodologia}. Diferente da revisão bibliográfica, que analisa obras de outros autores, a análise de documentação técnica foca em manuais, \textit{release notes} (notas de lançamento), especificações de API e guias oficiais de implementação. Isso elimina a subjetividade, permitindo avaliar as ferramentas com base no que elas foram projetadas tecnicamente para realizar.

Portanto, no contexto específico deste estudo, tal metodologia se justifica pela necessidade de validar as capacidades nativas de cada navegador através de suas definições oficiais. Consequentemente, a metodologia será utilizada para:

\begin{itemize}
    \item \textbf{Mapear Funcionalidades Oficiais}: Catalogar os recursos existentes em cada ambiente (Chrome, Firefox, Edge e Safari) consultando diretamente a documentação do desenvolvedor (ex: MDN Web Docs, Chrome Developers).
    \item \textbf{Realizar Análise Comparativa Técnica}: Confrontar as especificações das ferramentas para identificar discrepâncias de implementação, suporte a novos padrões CSS/JS e exclusividades de cada ecossistema.
    \item \textbf{Verificar Atualizações e Versionamento}: Analisar os registros de alteração (\textit{changelogs}) para compreender a evolução das ferramentas de depuração e o ciclo de introdução de novos recursos de produtividade.
\end{itemize}

Por fim, a aplicação dessa metodologia será realizada em etapas sequenciais: seleção das fontes oficiais, extração sistemática de dados técnicos, categorização das funcionalidades e, finalmente, a síntese comparativa dos resultados.